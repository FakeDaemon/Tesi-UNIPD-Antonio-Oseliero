% Acronyms
%\newacronym{api}{API}{Application Program Interface}
%\newacronym{sdk}{SDK}{Software Development Kit}

% Glossary
\newglossaryentry{api}{
    name={API},
    text={API},
    sort=api,
    description={In informatics, an API is a set of procedures available to programmers, typically grouped to form a toolkit for a specific task within a program. Its purpose is to provide an abstraction, usually between hardware and the programmer or between low-level and high-level software, simplifying the programming process}
}

\newglossaryentry{android}{
    name={Android},
    text={Android},
    sort=android,
    description={   
                    è un sistema operativo per dispositivi \textit{mobile} basato su una versione modificata del kernel Linux e altri 
                    \textit{software open-source}
                }
}

\newglossaryentry{csr}{
    name={CSR},
    text={CSR},
    sort=csr,
    description={   
                    Per Responsabilità Sociale delle Imprese (e delle organizzazioni) o secondo l'acronimo inglese CSR, 
                    \textit{Corporate Social Responsibility}, si intende l'integrazione su base volontaria, da parte delle imprese, 
                    delle preoccupazioni sociali e ambientali nelle loro operazioni interessate
                }
}

\newglossaryentry{erp}{
    name={ERP},
    text={ERP},
    sort=erp,
    description={
                    \textit{Enterprise Resource Planning}, è un tipo di sistema \textit{software} che aiuta le organizzazioni ad automatizzare e 
                    gestire i processi aziendali principali per ottenere le prestazioni ottimali. Il \textit{software} ERP coordina il flusso di dati 
                    tra i processi di un'azienda, fornendo un'unica fonte di informazioni e semplificando le operazioni nell'azienda. È in grado di 
                    collegare le attività finanziarie, della catena di approvvigionamento, delle operazioni, del commercio, dei \textit{report}, della 
                    produzione e delle risorse umane di un'azienda in una sola piattaforma
                }
}

\newglossaryentry{ide}{
    name={IDE},
    text={IDE},
    sort=ide,
    description={
                    \textit{Integrated Development Environment} o ambiente di sviluppo integrato, è un \textit{software} progettato per la 
                    realizzazione di applicazioni che aggrega strumenti di sviluppo comuni in un'unica interfaccia utente grafica. In genere è costituito da: 
                    editor del codice sorgente, strumenti che consentono di automatizzare la \textit{build} locale, un \textit{debugger} e 
                    strumenti per l'esecuzione di test automatici
                }
}

\newglossaryentry{ios}{
    name={iOS},
    text={iOS},
    sort=ios,
    description={
                    (\textit{iPhone Operating System}) è un sistema operativo sviluppato da Apple per i suoi dispositivi portabili come iPad 
                    e iPhone
                }
}

\newglossaryentry{macos}{
    name={macOS},
    text={macOS},
    sort=macos,
    description={
                    è il sistema operativo sviluppato da Apple per i propri \textit{computer} Macintosh
                }
}

\newglossaryentry{sdk}{
    name={SDK},
    text={Software Development Kit},
    sort=sdk,
    description={A Software Development Kit (SDK) is a collection of development tools in one installable package, facilitating application creation by providing a compiler, debugger, and sometimes a software framework. SDKs are typically specific to a hardware platform and operating system combination. Many application developers use specific SDKs to enable advanced functionalities such as advertisements, push notifications, etc}
}

\newglossaryentry{webapp}{
    name={Web App},
    text={web app},
    sort=webapp,
    description={
                    L'applicazione \textit{web}, o abbreviato \textit{web app}, nell'ambito dell'informatica e della programmazione, 
                    si riferisce alle applicazioni accessibili e fruibili attraverso il \textit{web}, quindi accessibili dall'utente 
                    tramite un \textit{browser web} con una connessione attiva
                }
}

\newglossaryentry{windows}{
    name={Windows},
    text={Windows},
    sort=windows,
    description={
                    Sistema operativo sviluppato da Microsoft Corporation. È uno dei sistemi operativi più utilizzati al mondo per \textit{personal computer}
                }
}

