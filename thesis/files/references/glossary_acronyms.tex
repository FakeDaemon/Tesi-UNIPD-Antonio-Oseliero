% Acronyms
%\newacronym{api}{API}{Application Program Interface}
%\newacronym{sdk}{SDK}{Software Development Kit}

% Glossary
\newglossaryentry{api}{
    name={API},
    text={API},
    sort=api,
    description={ \textit{Application Programming Interface} (interfaccia di programmazione delle applicazioni) sono un insieme di definizioni 
    e protocolli con i quali vengono realizzati e integrati \textit{software applicativi}. Le API stabiliscono il contenuto e la forma dei dati 
    necessari per la chiamata e quelli restituiti in risposta.\\
    {[\textit{fonte riportata in sitografia}]}}
}

\newglossaryentry{csr}{
    name={CSR},
    text={CSR},
    sort=csr,
    description={   
                    Per Responsabilità Sociale delle Imprese (e delle organizzazioni) o secondo l'acronimo inglese CSR, 
                    \textit{Corporate Social Responsibility}, si intende l'integrazione su base volontaria, da parte delle imprese, 
                    delle preoccupazioni sociali e ambientali nelle loro operazioni interessate.\\
                    {[\textit{fonte riportata in sitografia}]}}
}

\newglossaryentry{erp}{
    name={ERP},
    text={ERP},
    sort=erp,
    description={
                    \textit{Enterprise Resource Planning}, è un tipo di sistema \textit{software} che aiuta le organizzazioni ad automatizzare e 
                    gestire i processi aziendali principali per ottenere le prestazioni ottimali. Il \textit{software} ERP coordina il flusso di dati 
                    tra i processi di un'azienda, fornendo un'unica fonte di informazioni e semplificando le operazioni nell'azienda. È in grado di 
                    collegare le attività finanziarie, della catena di approvvigionamento, delle operazioni, del commercio, dei \textit{report}, della 
                    produzione e delle risorse umane di un'azienda in una sola piattaforma.\\
                    {[\textit{fonte della definizione inglese riportata in sitografia}]}}
}

\newglossaryentry{ide}{
    name={IDE},
    text={IDE},
    sort=ide,
    description={
                    \textit{Integrated Development Environment} o ambiente di sviluppo integrato, è un \textit{software} progettato per la 
                    realizzazione di applicazioni che aggrega strumenti di sviluppo comuni in un'unica interfaccia utente grafica. In genere è costituito da: 
                    \textit{editor} del codice sorgente, strumenti che consentono di automatizzare la \textit{build} locale, un \textit{debugger} e 
                    strumenti per l'esecuzione di test automatici.\\
                    {[\textit{fonte riportata in sitografia}]}}
}

\newglossaryentry{webapp}{
    name={Web App},
    text={web app},
    sort=webapp,
    description={
                    L'applicazione \textit{web}, o abbreviato \textit{web app}, nell'ambito dell'informatica e della programmazione, 
                    si riferisce alle applicazioni accessibili e fruibili attraverso il \textit{web}, quindi accessibili dall'utente 
                    tramite un \textit{browser web} con una connessione attiva
                }
}

