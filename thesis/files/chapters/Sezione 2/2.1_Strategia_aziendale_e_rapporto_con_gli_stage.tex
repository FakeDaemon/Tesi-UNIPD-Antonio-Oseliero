\section{Strategia aziendale e rapporto con gli \textit{stage}}\label{chap:strategia}
Come menzionato nel Capitolo \ref{chap:Propensione all'innovazione} per {\company} quella degli \textit{stage} è un occasione 
per innovarsi crescere dato che gli stagisti, che stanno per concludere il corso triennale di informatica, possiedono una 
buona conoscenza di quali sono le nuove tecnologie e hanno la curiosità di studiarle e implementarle.\\
L'azienda decide quindi di sfruttare l'occasione offerta dai tirocini per innovare e rinnovare i propri prodotti 
assegnando a gli stagisti progetti che devono essere: 
\textbf{utili allo studente} per crescere ed imparare, per questo 
si cerca di assegnare progetti non banali, che richiedono uno studio approfondito del codice e delle tecnologie e 
\textbf{basato su necessità reali}, ovvero derivato da 
necessità che l'azienda ha identificato mediante colloqui con i suoi clienti ed ha pertanto 
interesse ad implementare nel prodotto reale.\\
Una volta terminato il lavoro allo studente viene richiesto di esporre il suo lavoro agli sviluppatori che si occupano 
del prodotto in questione, in modo che possano valutare la qualità del lavoro svolto e capire la logica usata per lo sviluppo 
del progetto. Questo \textit{meeting} è fondamentale dato che gli sviluppatori e il \textit{project manager} possono 
vedere il progetto in funzione e valutare i benefici che una sua futura implementazione può portare al prodotto.\\
Quindi viene richiesto allo stagista di consegnare il codice sorgente e la documentazione tecnica in modo che possa essere usata per guidare 
l'eventuale implementazione futura.\\
Gli \textit{stage} per {\company} sono inoltre un modo per mettersi in contatto con studenti che si sono distinti in azienda e che hanno 
intenzione di interrompere il loro percorso universitario dopo la laurea, in modo da poter proporre loro un colloquio 
dove poterne valutare l'assunzione.