\section{Strategia aziendale e rapporto con gli \textit{stage}}\label{chap:strategia}
Come ho descritto nel capitolo \ref{chap:Propensione all'innovazione}, gli \textit{stage} rappresentano per {\company} 
un'opportunità di innovazione e crescita.\\
Gli stagisti, prossimi alla conclusione del corso triennale di laurea in informatica, 
possiedono una buona conoscenza di quali sono le nuove tecnologie e hanno la curiosità di studiarle e implementarle.\\
L'azienda decide quindi di sfruttare l'occasione offerta dai tirocini per innovare e rinnovare i propri prodotti.\\
I progetti assegnati agli stagisti devono essere formativi, richiedendo uno studio approfondito del codice e delle tecnologie. 
Questi compiti non devono essere banali e devono mirare a stimolare la crescita e l'apprendimento dello studente. 
Inoltre, devono basarsi su necessità reali, identificate mediante colloqui con i clienti, e sono pertanto di interesse per 
{\company} nell'ottica di una futura implementazione dei progetti nei prodotti reali.\\
Al termine del lavoro, lo studente presenta il suo operato agli sviluppatori e al \textit{project manager} del prodotto 
in questione. Questo \textit{meeting} è cruciale per valutare la qualità del lavoro svolto e comprendere la logica di 
sviluppo adottata. Permette inoltre agli sviluppatori e al \textit{project manager} di esaminare il progetto in funzione e 
valutare i benefici di una sua futura implementazione.\\
Successivamente, si richiede allo stagista di consegnare il codice sorgente e la documentazione tecnica, che serviranno 
da guida per l'eventuale implementazione futura.\\
Gli \textit{stage} per {\company} sono inoltre un modo per mettersi in contatto con studenti che si sono distinti in azienda 
e che hanno intenzione di interrompere il loro percorso universitario dopo la laurea, in modo da poter proporre loro un colloquio 
dove poterne valutare l'assunzione.