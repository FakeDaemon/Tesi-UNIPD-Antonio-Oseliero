\section{Scelta dell'attività di \textit{stage}}
Sono entrato in contatto con l'azienda all'evento StageIT 2024, organizzato dall'Università di Padova e promosso Confindustria 
Veneto Est. Questo evento permette alle aziende di incontrare gli studenti e di avere un breve colloquio in cui poter 
descrivere alcune caratteristiche dell'azienda e i progetti offerti per lo \textit{stage}.\\
Qui ho avuto la possibilità di incontrare molte realtà che operano in settori distinti, dallo sviluppo \textit{web} a progetti in 
ambito \textit{cyber security}, non limitandomi a cercare progetti che rispecchiassero le mie conoscenze pregresse, ma esplorando tutte le 
varie opzioni.\\
Incontrando {\company} all'evento ho avuto modo di conoscere l'azienda e di richiedere approfondimenti riguardo i loro progetti di 
\textit{stage}, il cui elenco e breve descrizione erano stati diffusi prima dell'evento per tutte le aziende coinvolte.\\
Dopo l'evento, ho selezionato i progetti che più di tutti avevano attirato la mia attenzione e fissato un ulteriore colloquio con 
le aziende per discutere in maniera più approfondita dei progetti proposti, in particolare i colloqui hanno avuto come argomenti 
principali: il \textbf{progetto} in maniera ancora più approfondita, le \textbf{tecnologie} utilizzate, \textbf{l'azienda} e una discussione ad alto livello 
su \textbf{come questo progetto sarebbe potuto essere implementato}.\\
Durante l'incontro con {\company} ho inoltre potuto fare un \textit{tour} dell'azienda e conoscere i dipendenti, oltre a poter 
vedere il luogo in cui lavorano e come interagiscono con i clienti.\\
La decisione di lavorare al progetto "Modulo agenti" nasce da diversi fattori:\\
In primo luogo perché mi ha permesso di lavorare ad un'applicazione Android, ambito che mi interessa molto, con tecnologie 
moderne e ricercate in ambito lavorativo come React Native, tecnologia per me non del tutto nuova visto l'esperienza 
pregressa con React che non include però le funzionalità specifiche per applicazioni \textit{mobile}, e ASP.NET Core per 
lo sviluppo di \gls{api}.\\
In secondo luogo il progetto offerto da {\company} mi ha permesso di lavorare sia come sviluppatore \textit{front-end} 
creando le interfacce per l'\textit{app} che come sviluppatore \textit{back-end} creando e modificando le \gls{api} dell'applicazione. 
