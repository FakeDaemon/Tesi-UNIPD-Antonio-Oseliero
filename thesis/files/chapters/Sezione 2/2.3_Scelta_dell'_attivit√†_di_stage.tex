\section{Scelta dell'attività di \textit{stage}}
L'incontro con {\company} è avvenuto durante l'evento StageIT 2024, organizzato dall'Università di Padova.\\
Questo evento offre alle aziende l'opportunità di incontrare gli studenti e condurre brevi colloqui, illustrando le 
caratteristiche dell'azienda e i progetti offerti per lo \textit{stage}.\\
In questa occasione, ho avuto l'opportunità di esplorare diverse realtà operanti in settori distinti, dallo sviluppo 
\textit{web} a progetti in ambito \textit{cyber security}. Non mi sono limitato a cercare progetti allineati alle mie 
conoscenze pregresse, ma ho esplorato diverse opzioni.\\
Durante l'incontro con {\company}, ho approfondito la conoscenza dell'azienda e mi sono state fornite ulteriori informazioni 
riguardo i loro progetti di \textit{stage}, descritti in un elenco diffuso prima dell'evento per tutte le 
aziende coinvolte.\\
Successivamente ho selezionato i progetti più interessanti e fissato ulteriori colloqui con
le aziende per discutere in maniera più approfondita delle proposte. I colloqui si sono focalizzati principalmente su: 
il progetto in dettaglio, le tecnologie utilizzate, l'azienda e una discussione ad alto livello
sulle possibili implementazioni del progetto.\\
Durante l'incontro con {\company} ho inoltre potuto fare un \textit{tour} dell'azienda, che mi ha permesso di conoscere i 
dipendenti e osservare il loro ambiente di lavoro e le interazioni con i clienti.\\
La scelta di lavorare al progetto "Modulo agenti" è stata motivata da diversi fattori:
innanzitutto, mi ha consentito di lavorare ad un'applicazione Android, ambito di grande interesse personale, utilizzando tecnologie
moderne e ricercate come React Native. Questa tecnologia non mi è del tutto estranea, grazie all'esperienza
pregressa con React, sebbene non includa le funzionalità specifiche per applicazioni \textit{mobile}. 
Il progetto ha previsto anche l'utilizzo di ASP.NET Core per lo sviluppo di \gls{api}.\\
In secondo luogo, l'opportunità offerta da {\company} mi ha permesso di lavorare sia come sviluppatore \textit{front-end},
creando le interfacce per l'\textit{app}, che come sviluppatore \textit{back-end}, creando e modificando le 
\gls{api} dell'applicazione.