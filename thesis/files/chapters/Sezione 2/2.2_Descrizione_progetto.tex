\section{Descrizione progetto}
{\movi} come descritto nel capitolo \ref{chap:Tecnologie} è un'applicazione che le aziende forniscono ai loro clienti 
permettendogli di visualizzare il catalogo di prodotti offerti dall'azienda e di creare quindi gli ordini per il \textit{restock} 
della merce. Questa applicazione permette di \textbf{vedere lo storico degli ordini} effettuati dal cliente, \textbf{visualizzare il 
set tipico} (ovvero un ordine predefinito creato inserendo tutti i prodotti solitamente acquistati dal cliente nelle rispettive quantità), 
\textbf{creare un ordine} con i prodotti offerti dal fornitore, visualizzare un \textbf{carrello} di riepilogo dell'ordine e uno \textbf{storico} che 
contiene tutti i documenti generati a seguito della conferma dell'ordine.\\
Quello che richiede il progetto assegnatomi da {\company} è di sviluppare un modulo per {\movi} chiamato "Modulo Agenti", ovvero 
l'insieme di interfacce, funzioni, \gls{api}, ecc. che permette l'autenticazione di un nuovo tipo di utente, ovvero gli agenti aziendali, 
che selezionando da una lista uno dei clienti a lui assegnati permette di operare nell'app come il cliente selezionato.\\
La necessità di implementare questo modulo nasce da alcuni bisogni segnalati da i suoi clienti a {\company} e che 
mi sono stati descritti durante il primo incontro in azienda con il tutor aziendale, i principali motivi sono:
\begin{itemize}
    \item \textbf{MoviSELL, l'app pensata per gli agenti aziendali, è disponibile solo per tablet iOS}. Questo può costituire un problema 
          per alcune aziende che, per dotare i propri agenti dell'app, sono obbligate ad acquistare questi tablet per i propri agenti. 
          Per aziende con agenti plurimandatari, cioè che rappresentano più aziende contemporaneamente, questo requisito si 
          rivela essere particolarmente oneroso da soddisfare dato che si richiede di fornire i tablet non ai propri dipendenti, 
          ma a professionisti esterni;
    \item \textbf{Non tutti gli agenti si trovano a loro agio ad usare il tablet}, trovandolo ingombrante e scomodo, soprattutto per chi 
          lavora molto in mobilità. Pertanto, avere un'alternativa per smartphone risulta preferibile.
    \item MoviSELL è un'app ricca di funzionalità, ma può risultare di difficile utilizzo per chi non ha dimestichezza con gli 
          strumenti digitali. \textbf{{\movi} risulta molto più semplice e intuitiva rimuovendo la barriera tecnologica per alcuni 
          agenti} e permettendo loro di svolgere il loro lavoro;
    \item Alcuni clienti preferiscono contattare direttamente gli agenti per effettuare i loro ordini, invece di usare {\movi}. 
          \textbf{Il modulo agenti semplifica l'operazione di creazione dell'ordine per gli agenti}, evitando loro di dover appuntare 
          la merce da ordinare e poi effettuare l'ordine dal computer una volta rientrati in ufficio.
\end{itemize}