\section{Strategia aziendale e rapporto con gli stage}
Come menzionato nel Capitolo \ref{chap:Propensione all'innovazione} per {\company} quella degli stage è un occasione 
per innovarsi crescere. Gli stagisti, che stanno per concludere il corso triennale di informatica, hanno una 
buona conoscenza di quali sono le nuove tecnologie e hanno la curiosità di studiarle e implementarle.
Ecco perché quindi, nonostante esistano dei vincoli tecnologici (vedi capitolo TODO), {\company} apprezza 
l'indipendenza dello stagista nell'introdurre nuove tecnologie nel progetto dato che gli permette di studiare i 
benefici che questa comporta e valutarne l'implementazione nel prodotto reale.\\

L'azienda adotta una strategia che gli permette di sfruttare il lavoro dello stagista come risorsa 
utile per i prodotti che commercializzano: prima verificano quali progetti possono offrire agli stagisti, questi 
non possono essere troppo complessi ma nemmeno banali e devono nascere da bisogni concreti. Quindi una volta 
che il progetto è stato completato l'azienda riunisce il team di sviluppatori interessati al progetto 
e lo stagista per discutere come lo studente ha risolto i bisogni dell'azienda contenuti descrizione del progetto.
Infine raccoglie tutto il lavoro dello stagista per studiarlo ed implementare quindi le nuove modifiche nel prodotto 
reale.
Nel mio caso il progetto di stage era di particolare interesse per l'azienda in quanto è attualmente in 
sviluppo la nuova versione dell'app {\movi} che vedrà l'implementazione del modulo agenti come prossimo 
passo dopo il completamento della nuova versione.

\section{Strategia aziendale e rapporto con gli stage}

Come menzionato nel Capitolo \ref{chap:Propensione all'innovazione}, per {\company} gli stage rappresentano 
un'opportunità per innovarsi e crescere. Gli stagisti, che stanno per concludere il corso triennale di 
informatica, possiedono una buona conoscenza delle nuove tecnologie e la curiosità di studiarle e 
implementarle. Nonostante i vincoli tecnologici (vedi Capitolo TODO), {\company} apprezza 
l'indipendenza dello stagista nell'introdurre nuove tecnologie nei progetti, permettendo così di 
valutare i benefici di queste innovazioni e considerarne l'implementazione nei prodotti reali.

L'azienda adotta una strategia che le consente di sfruttare il lavoro degli stagisti come risorsa utile per i prodotti commercializzati. 
Prima di tutto, vengono individuati i progetti da offrire agli stagisti: questi non devono essere né troppo complessi né troppo banali, 
ma devono rispondere a esigenze concrete. Una volta completato il progetto, l'azienda riunisce il team di sviluppatori interessati 
al progetto e lo stagista per discutere come lo studente ha risolto le esigenze aziendali descritte nel progetto. 
Infine, l'azienda raccoglie tutto il lavoro dello stagista per studiarlo e implementare le modifiche nel prodotto reale.

Nel mio caso, il progetto di stage era di particolare interesse per l'azienda, poiché è attualmente in sviluppo la nuova versione dell'
app {\movi}, che vedrà l'implementazione del modulo agenti come prossimo passo dopo il completamento della nuova versione.
