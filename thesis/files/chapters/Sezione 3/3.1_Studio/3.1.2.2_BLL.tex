\subsubsection{\textit{Business logic layer e service pattern}}
Il \textit{business layer} in {\movi} implementa il \textit{service pattern}.
Il \textit{service pattern} è un modello di progettazione \textit{software} che separa la logica di \textit{business} dell'
applicazione dal resto del sistema.\\
Nel contesto di {\movi}, il Il \textit{service pattern} è implementato nel \textit{business layer} 
e funge da intermediario tra il \textit{layer} di presentazione (\textit{API Controllers}) e il \textit{data access layer}.
Il \textit{service pattern} è implementato principalmente attraverso un componente chiave: \texttt{BaseService<TContext, TEntity>} 
che gestisce il flusso di dati, definisce le operazioni CRUD e incapsula la logica dei servizi.\\
I servizi sono classi concrete che estendono \texttt{BaseService} ed implementano la logica di \textit{business} 
e vengono richiamati dagli \textit{API Controllers} del livello superiore.
Utilizzando questo \textit{pattern} i servizi nascondono la logica e rendono modulare il servizio, implementano inoltre la gestione 
degli errori e permettono di mappare i dati dello strato precedente in DTO (vedi capitolo \ref{chap:dto}).
L'implementazione del \textit{service pattern} offre diversi vantaggi:
\begin{itemize}
\item \textbf{Separazione delle Responsabilità}: La logica di \textit{business} è chiaramente separata dalla logica di 
      presentazione e di accesso ai dati.
\item \textbf{Riusabilità}: La logica di \textit{business} incapsulata nei servizi può essere facilmente riutilizzata 
      in diverse parti dell'applicazione.
\item \textbf{Manutenibilità}: La struttura modulare facilita la manutenzione e l'estensione del codice.
\item \textbf{Testabilità}: L'uso di interfacce e iniezione delle dipendenze rende il codice altamente testabile.
\end{itemize}
In questo \textit{layer} vengono implementati anche meccanismi di sicurezza, tramite la definizione di un \textit{token} 
che viene creato al momento dell'autenticazione dell'utente e gli viene passato come risposta dalla \gls{api} di \textit{login} 
e contiene criptate al suo interno alcune informazioni riguardo l'utente.\\
In conclusione, il \textit{business layer} e il \textit{service pattern} in {\movi} forniscono una robusta struttura 
per l'implementazione della logica di \textit{business}, agendo come un ponte efficace tra il layer di presentazione e il 
\textit{layer} di accesso ai dati, garantendo al contempo modularità, riusabilità e manutenibilità del codice.