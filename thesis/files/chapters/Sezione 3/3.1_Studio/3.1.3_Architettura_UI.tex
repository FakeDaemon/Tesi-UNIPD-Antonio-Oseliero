\subsection{Architettura Front-end}
                                          TODO SCRIVERE MEGLIO E GRAFICO
Infine ho studiato l'architettura del \textit{front-end}.\\
{\movi} come la maggior parte delle applicazioni React è strutturata secondo un'architettura chiamata \textit{component 
based}, che favorisce la modularità e il riutilizzo del codice.\\
Per componente in questo contesto intendiamo un gruppo di funzionalità correlate che risiede dietro un'interfaccia ben definita.
L'organizzazione del codice in questa architettura viene strutturato dividendo le varie parti per aree tematiche, 
migliorando la manutenibilità e la scalabilità del progetto.\\
La \textit{repository} è suddivisa nelle seguenti cartelle:
\begin{itemize}
    \item \textbf{Android}: specifica per React Native, contiene i file di configurazione 
          per la versione Android dell'\textit{app} e \textit{file} come \texttt{build.gradle} e \texttt{AndroidManifest.xml} 
          per la configurazione del compilatore Android;
    \item \textbf{iOS}: specifica per React Native, contiene il progetto Xcode e i file di configurazione 
          per la versione iOS dell'\textit{app} e \textit{file} per la configurazione del compilatore di iOS;
    \item \textbf{lang}: contiene i \textit{file} \texttt{en.json e it.json} che contiene il testo usato nell'applicazione 
          in italiano e inglese;
    \item \textbf{src}: contiene tutto il codice del \textit{front-end} e realizza l'architettura \textit{component 
          based} che sarà descritta in seguito;
    \item \textbf{\textit{file} di varia natura nella root della \textit{repository}}: questi \textit{file} sono fondamentali 
          per garantire un corretto funzionamento e una gestione efficiente del progetto, qui riporto quelli più importanti:
          \begin{itemize}
            \item \textbf{\texttt{package.json}}: contiene le informazioni sul progetto come \textbf{nome del progetto, versione,
                  script di comando e dipendenze}. È il file principale per gestire i pacchetti Node.js utilizzati nel progetto;
            \item \textbf{\texttt{yarn.lock}}: questo file viene generato automaticamente quando si genera il progetto e 
                  permette di gestire ed installare le dipendenze del progetto con le loro versioni esatte;
            \item \textbf{\texttt{metro.config.js}}: configurazione di Metro, il \textit{bundler} di JavaScript predefinito 
                  per React Native. Questo file permette di personalizzare il comportamento di Metro, come 
                  l'aggiunta di alias, path customizzati, o l'esclusione di determinati file dalla build.
            \item \textbf{\texttt{babel.config.js}} babel.config.js: Configura Babel, il transpiler di JavaScript. 
                  Definisce come il codice deve essere trasformato per essere compatibile con le varie versioni di JavaScript e i 
                  diversi ambienti in cui verrà eseguito.
            \item \textbf{\texttt{index.js}} index.js: Punto d'ingresso principale dell'applicazione React Native. Qui viene 
                  avviato il rendering del componente radice dell'\textit{app}, generalmente importando e registrando il componente 
                  principale.
            \item app.tsx: Contiene il componente radice dell'applicazione. Qui vengono definiti l'interfaccia utente principale 
                  e la logica di base dell'\textit{app}.
            \item tsconfig.json: Configura il compilatore TypeScript, specificando le opzioni di compilazione e il comportamento del transpiling del codice TypeScript in JavaScript.
          \end{itemize}
\end{itemize}

Descriviamo ora ogni area della cartella src,
\begin{itemize}
      \item Cartella components": Contiene componenti riutilizzabili e generici, suddivisi ulteriormente in sottocartelle come "ui" per elementi dell'interfaccia utente e "layout" per strutture di pagina comuni.
      \item Cartella "pages" o "views": Ospita componenti che rappresentano intere pagine o viste principali dell'applicazione.
      \item Cartella "services": Contiene moduli per la gestione delle chiamate API e la logica di business condivisa.
      \item Cartella "hooks": Raccoglie hook personalizzati per la gestione dello stato e della logica riutilizzabile.
      \item Cartella "context" o "store": Dedicata alla gestione dello stato globale dell'applicazione, sia utilizzando il Context API di React che soluzioni come Redux.
      \item Cartella "utils" o "helpers": Per funzioni di utilità e helper condivisi in tutta l'applicazione.
      \item Cartella "assets": Per risorse statiche come immagini, font e file di stile globali.
\end{itemize}
Questa organizzazione per aree facilita la navigazione nel codice e promuove una chiara separazione delle responsabilità. 
React utilizza un flusso di dati unidirezionale, dove le informazioni fluiscono dall'alto verso il basso attraverso le 
proprietà (props), mentre lo stato dell'applicazione è gestito attraverso soluzioni di gestione dello stato.
