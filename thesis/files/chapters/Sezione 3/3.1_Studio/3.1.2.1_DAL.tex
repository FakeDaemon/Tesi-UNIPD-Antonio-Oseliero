\subsubsection{\textit{Data access layer e repository pattern}}
Il \textit{data access layer} di {\movi} è implementato seguendo il \textit{Repository Pattern}, un modello di progettazione 
che separa la logica di accesso ai dati dal resto dell'applicazione. Questo \textit{pattern} crea un'astrazione tra il livello 
di accesso ai dati e la logica di \textit{business}, consentendo una gestione più flessibile e manutenibile dei dati.\\
Il \textit{repository pattern} è implementato attraverso le classi:
\begin{itemize}
\item \texttt{\textbf{ReadOnlyRepository<TContext>}}: Questa classe astratta fornisce metodi per operazioni di sola lettura sul database.
\item \texttt{\textbf{Repository<TContext>}}: Estende \texttt{ReadOnlyRepository} aggiungendo metodi per operazioni di scrittura.
\end{itemize}
Qui vengono definiti anche i modelli e le \textit{migration} generate tramite Entity Framework, un 
\textit{mapper} ad alto livello integrabile a .NET che permette la trasposizione delle tabelle del \textit{database} in 
classi del dominio applicativo.\\
Entity Framework, originariamente parte del \textit{framework} .NET ma ora distribuito come pacchetto indipendente 
installabile attraverso l'\textit{installer} di Visual Studio, promuove un approccio di sviluppo \textit{code first}. 
Entity Framework permette infatti di gestire il \textit{database} attraverso i modelli, ovvero delle particolari classi 
che descrivono la forma delle tabelle e i loro attributi.\\
Creando o modificando questi modelli è possibile creare o modificare la tabella della base dati senza la necessità di 
interventi manuali diretti, ma attraverso le \textit{migration} generate dal \textit{framework} automaticamente. 
Quando si apportano modifiche al modello, Entity Framework compara le \textit{migration} esistenti, determinando così 
lo stato attuale del \textit{database}. Questo processo permette di identificare precisamente le modifiche necessarie 
alla struttura del \textit{database}. Se il processo di analisi e generazione va a buon fine, Entity Framework crea 
una nuova \textit{migration} che riporta tutte le modifiche applicate. Questo meccanismo assicura una gestione 
coerente e tracciabile dell'evoluzione della struttura del \textit{database}, mantenendo sincronizzati il modello dei 
dati nell'applicazione e la struttura effettiva del \textit{database}.\\
In questo caso il \textit{template} \texttt{TContext} delle classi \texttt{Repository} e \texttt{ReadOnlyRepository} 
rappresenta \texttt{DbContext}, una classe fondamentale in Entity Framework che rappresenta una sessione con il 
\textit{database}. Essa permette di:
\begin{itemize}
      \item \textbf{Eseguire query sul database};
      \item \textbf{Tracciare le modifiche apportate alle entità};
      \item \textbf{Persistere i cambiamenti nel database}.
\end{itemize}


