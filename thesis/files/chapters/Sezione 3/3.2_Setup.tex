\section{\textit{Setup}}
Sempre durante la prima fase di studio, ho avviato il \textit{setup} dell'ambiente di sviluppo. Il mio obiettivo era 
di riuscire a configurare i diversi \textit{editor} ed eseguire il progetto in locale con il \textit{database} di prova 
assegnatomi.\\
La configurazione del \textit{database} in locale e la creazione di un utente di accesso per poter modificare le tabelle 
con SSMS si sono rivelate relativamente semplici, grazie l'aiuto degli sviluppatori. In meno di due ore il \textit{database} 
era disponibile in locale sulla porta 2022.\\
L'avvio delle \gls{api} in locale ha richiesto più tempo, ma sempre grazie al supporto del \textit{team} di {\company} sono 
riuscito in poco tempo a completare il \textit{setup} in tempi ragionevoli.
Inizialmente ho configurato Visual Studio e risolto alcuni problemi di compatibilità dei pacchetti e versione di .NET.
Quindi ho modificato gli \textit{endpoint} di connessione al \textit{database} e quelli in cui esporre le \gls{api}, 
impostando come porta locale la 4048.\\
Questo compito è stato leggermente più complesso del precedente, poiché per evitare errori imprevisti ho dovuto limitare 
la possibilità di ricercare il \textit{company database}.
Come ho descritto nel capitolo \ref{chap:struttura database} il \textit{company database}, a differenza del \textit{common e system 
database}, non è statico ma scelto dinamicamente a seconda del valore dell'attributo \texttt{CompanyCode} dell'utente durante la 
fase di \textit{login}. Con le modifiche apportate la ricerca del \textit{company database} restituisce un unico risultato: 
\texttt{localhost:2022}.\\
Quando si avviano le \gls{api} e si digita \texttt{localhost:4048} nella barra degli indirizzi del \textit{browser}, viene 
visualizzata l'interfaccia di Swagger. Questa permette di testare il corretto funzionamento delle \gls{api} e la loro 
integrazione con il \textit{database}.\\
La parte che ha richiesto più tempo, ritardando di alcuni giorni la fase di modifica dell'\gls{api} di \textit{login}, è stata 
l'avvio in locale del \textit{front-end}. Questa attività ha richiesto diverso tempo a causa di vari errori che occorrevano 
durante l'avvio e del conseguente \textit{troubleshooting} necessario per risolverli.\\
I problemi principali erano dovuti all'Android SDK, il \textit{kit} di sviluppo \textit{software} per Android. Nonostante fosse 
già installato sul \textit{computer}, le variabili d'ambiente non erano configurate. Dopo aver sistemato le variabili 
d'ambiente, ho scoperto che ADB (\textit{Android Debug Bridge}), uno strumento fondamentale di Android SDK, non risultava 
installato o utilizzabile, e ho dovuto quindi installarlo e configurarlo separatamente.\\
Successivamente ho cambiato gli \textit{endpoint} per consentire la connessione in locale alle \gls{api} ed ho provato ad installare 
l'\textit{app} nel AVD (\textit{Android Virtual Device}). AVD è lo strumento offerto da Android Studio per simulare un dispositivo 
Android su \textit{computer}, senza utilizzare un dispositivo reale. I limiti di questo approccio sono subito evidenti a 
causa dell'eccessiva pesantezza dello strumento che richiede prestazioni molto elevate. 
Ho quindi optato per l'utilizzo di uno \textit{smartphone} fornito dall'azienda, sostituendo Android Studio con Visual Studio Code.\\
Per poter lavorare in locale con lo \textit{smartphone}, ho dovuto collegarlo al \textit{computer} e utilizzare il comando 
\texttt{abd reverse} dell'Android SDK. Questo comando permette di utilizzare le porte locali del dispositivo come se fossero quelle 
del \textit{computer}. Ad esempio lanciando il comando \texttt{adb -s [nome del dispositivo] reverse tcp:4048 tcp:4048} e 
avviando l'applicazione dallo \textit{smartphone}, al momento della connessione alle \gls{api} locali (esposte sulla porta 
locale 4048 del \textit{computer}), il dispositivo utilizzerà la porta del \textit{pc} anziché la sua porta 
locale 4048.\\
Quest'ultima attività ha richiesto qualche giorno di lavoro, posticipando l'inizio della seconda fase.
Tuttavia, al termine di questo processo, l'applicazione è stata completamente configurata per operare in locale, 
permettendo così l'inizio dello sviluppo vero e proprio.


