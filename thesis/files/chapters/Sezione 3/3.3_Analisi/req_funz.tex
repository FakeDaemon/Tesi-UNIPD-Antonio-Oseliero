\begin{center}
    \rowcolors{1}{}{tableGray}
    \begin{longtable}{|p{2.25cm}|p{7.75cm}|p{2.25cm}|}
    \hline
    %\rowcolor{hyperColor!5}
    \multicolumn{1}{|c|}{\textbf{Requisito}} & \multicolumn{1}{c|}{\textbf{Descrizione}} & \multicolumn{1}{c|}{\textbf{Fonte}}\\
    \hline 
    \endfirsthead
    \rowcolor{white}
    \multicolumn{3}{c}{{\bfseries \tablename\ \thetable{} -- Continuo della tabella}}\\
    \hline
    %\rowcolor{hyperColor!5}
    \multicolumn{1}{|c|}{\textbf{Requisito}} & \multicolumn{1}{c|}{\textbf{Descrizione}} & \multicolumn{1}{c|}{\textbf{Fonte}}\\
    \hline 
    \endhead
    \hline
    \rowcolor{white}
    \multicolumn{3}{|r|}{{Continua nella prossima pagina...}}\\
    \hline
    \endfoot
    \endlastfoot
    
    FO1 & Un utente deve poter effettuare il \textit{login} ed essere automaticamente riconosciuto come cliente & UC 1.1, Capitolato \\
    FO1.1 & Un cliente deve essere spostato nella \texttt{Homepage} in seguito al \textit{login} & UC 1.1, Capitolato \\
    FO1.2 & Un cliente non deve poter notare nessuna differenza rispetto alla precedente versione dell'\textit{app} & Capitolato \\
    FO2 & Un utente deve poter effettuare il \textit{login} ed essere automaticamente riconosciuto come agente & UC 1.2, Capitolato \\
    FO2.1 & Un agente deve essere spostato nella \texttt{Homepage Agenti} in seguito al \textit{login} & UC 1.2, Studente \\
    FO3 & Un utente deve visualizzare un messaggio d'errore se le credenziali sono errate & UC 1.3, Studente \\
    FO4 & Un agente deve poter visualizzare una lista con i suoi clienti & Capitolato \\
    FD4.1 & Un agente deve poter ricercare un cliente all'interno della lista dei suoi clienti & UC 2.1.1, \textit{Tutor} \\
    FO4.2 & Un agente deve poter navigare la lista dei suoi clienti & UC 2.1.2, Capitolato \\
    FO4.3 & Un agente deve poter selezionare dalla lista uno dei suoi clienti & UC 2.1.3, Capitolato \\
    FO4.4 & Un agente deve visualizzare un messaggio d'errore che riporta la frase "nessun cliente trovato" se la lista clienti è vuota & UC 2.4, \textit{Tutor} \\
    FD4.5 & Un agente deve visualizzare un messaggio d'errore che riporta la frase "nessun cliente trovato" se la ricerca clienti non ha prodotto risultati & UC 2.1.4, \textit{Tutor} \\
    FO4.6 & Un agente deve essere spostato nella pagina \texttt{Homepage} dopo aver selezionato un cliente dalla lista & UC 2.1.3, Studente \\
    FO4.7 & Un agente deve essere autenticato come il cliente selezionato dopo aver selezionato un cliente dalla lista & UC 2.1.3, Capitolato \\
    FD5 & Un agente deve poter visualizzare un menu nella \texttt{Homepage Agenti} & Studente \\
    FD5.1 & Un agente deve poter premere il pulsante "Impostazioni" dal menu nella \texttt{Homepage Agenti} & UC 2.2, Studente \\
    FD5.2 & Un agente deve essere spostato nella pagina \texttt{Impostazioni} dopo aver premuto il pulsante "Impostazioni" & UC 2.2, Studente \\
    FD5.3 & Un agente deve poter premere il pulsante "\textit{Account}" dal menu nella \texttt{Homepage Agenti} & UC 2.3, Studente \\
    FD5.4 & Un agente deve essere spostato nella pagina \texttt{Account} dopo aver premuto il pulsante "\textit{Account}" & UC 2.3, Studente \\
    FO6 & Un agente autenticato come cliente deve poter visualizzare un menu nella \texttt{Homepage} & Studente \\
    FD6.1 & Un agente autenticato come cliente deve poter ritornare alla \texttt{Homepage Agenti} premendo 
            il pulsante "\textit{Homepage} Agenti" nel menu della \texttt{Homepage} & 3.1, Studente \\
    FD6.2 & Un agente autenticato come cliente deve essere spostato nella pagina \texttt{Prodotti} premendo 
            il pulsante "Prodotti" nel menu della \texttt{Homepage} & 3.2, Capitolato \\
    FD6.2.1 & Un agente autenticato come cliente deve poter operare come il cliente selezionato nella pagina \texttt{Prodotti} & 3.2, Capitolato \\
    FD6.3 & Un agente autenticato come cliente deve essere spostato nella pagina \texttt{Carrello} premendo 
            il pulsante "Carrello" nel menu della \texttt{Homepage} & 3.3, Capitolato \\
    FD6.3.1 & Un agente autenticato come cliente deve poter operare come il cliente selezionato nella pagina \texttt{Carrello} & 3.3, Capitolato \\
    FD6.4 & Un agente autenticato come cliente deve essere spostato nella pagina \texttt{Storico} premendo 
            il pulsante "Storico" nel menu della \texttt{Homepage} & 3.4, Capitolato \\
    FD6.4.1 & Un agente autenticato come cliente deve poter operare come il cliente selezionato nella pagina \texttt{Storico} & 3.4, Capitolato \\
    FD7.1 & Un agente deve poter modificare il tema impostando il tema chiaro dalla pagina \texttt{Impostazioni} & UC 4, \textit{Tutor} \\
    FD7.2 & Un agente deve poter modificare il tema impostando il tema scuro dalla pagina \texttt{Impostazioni} & UC 4, \textit{Tutor} \\
    FO7.3 & Un agente non deve poter modificare la propria schermata d'avvio dalla pagina \texttt{Impostazioni} & Studente \\
    FD8 & Un agente deve poter effettuare il \textit{logout} dalla pagina \texttt{Account} & UC 5, \textit{Tutor} \\
    QO1 & Insieme al modulo deve essere consegnato anche un manuale tecnico & Capitolato, \textit{Tutor} \\
    QO2 & Insieme al modulo deve essere consegnato anche un manuale utente & Capitolato, \textit{Tutor} \\
    UO1 & Un agente deve visualizzare il suo nome all'interno della schermata \texttt{Homepage Agenti} & Capitolato, Studente \\
    UO2 & Un agente deve visualizzare il nome del cliente selezionato all'interno della schermata \texttt{Homepage} & Capitolato, Studente \\
    UO3 & Ogni voce della lista deve riportare alcune informazioni del cliente & \textit{Tutor} \\
    UO3.1 & Ogni voce della lista deve riportare il nome del cliente & \textit{Tutor} \\
    UO3.2 & Ogni voce della lista deve riportare l'indirizzo del cliente & \textit{Tutor} \\
    UD4 & L'applicazione deve essere disponibile nella lingua italiana & \textit{Tutor} \\
    UD5 & L'applicazione deve essere disponibile nella lingua inglese & \textit{Tutor} \\
    UD6 & Deve essere possibile ricercare un cliente nelle lista attraverso l'uso di una \textit{search bar} & \textit{Tutor} \\
    UD7 & Durante la digitazione del parametro di ricerca i risultati devono essere filtrati anche per i parametri parziali & \textit{Tutor} \\
    UD8 & Durante la ricerca i clienti devono essere filtrati secondo il parametro digitato dall'agente & \textit{Tutor} \\
    UD9 & Lo stile dell'applicazione deve essere compatibile con dispositivi \textit{tablet} & Capitolato, \textit{Tutor} \\
    VO1 & Il modulo deve utilizzare i \textit{database} di {\movi} apportando eventuali modifiche alla struttura & Capitolato, \textit{Tutor} \\
    VO2 & Il modulo deve estendere le \gls{api} esistenti sviluppate in .NET & Capitolato, \textit{Tutor} \\
    VO3 & Il modulo deve estendere le interfacce esistenti sviluppate in React Native & Capitolato, \textit{Tutor} \\
    \hline
    \hiderowcolors
    \caption{Tabella del tracciamento dei requisiti funzionali e non funzionali.}
    \label{tab:requisiti_funzionali}
    \end{longtable}
\end{center}