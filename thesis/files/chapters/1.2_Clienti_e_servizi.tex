\section{Clienti e servizi}
{\company} ha come clienti piccole e medie imprese situate in prevalenza in Veneto, Emilia-Romagna e in generale nel Nord Italia, possiamo trovare però 
anche clienti dal Centro Italia e dalla Sardegna. \\
Il gestionale che propone può adattarsi a qualsiasi tipo di azienda indipendentemente dal settore in cui 
operi (anche se come vedremo vengono venduti dei gestionali ad hoc per i settori: \textbf{petrolifero, ittico, assistenza post-vendita, ortofrutticolo, antincendi 
e antinfortunistica, trasporti}).\\ Una volta implementato il gestionale all'interno dell'azienda del cliente viene offerta una formazione all'utilizzo del 
\textit{software} per i dipendenti, che parteciperanno a delle riunioni tenute da un consulente tecnico che ne illustrerà le funzionalità e insegnerà come sfruttarle 
al meglio.\\
{\company} propone due linee di prodotti principali: \textbf{Vision} e \textbf{MoviDAT}.
La prima, Vision, è la linea di gestionali dell'azienda, con VisionENTERPRISE, che è il loro \gls{erp} di punta, 
e poi una serie di soluzioni verticali per venire incontro alle specifiche esigenze delle varie aziende con cui {\company} opera.
Ognuna delle soluzioni verticali offerte dall'azienda è una variazione di VisionENTERPRISE, che viene arricchita con 
funzionalità specifiche per adattarsi a specifici settori. In particolare quindi nella linea Vision abbiamo:
\begin{itemize}
      \item \textbf{VisionENTERPRISE}, \gls{erp} di punta dell'azienda e dedicato ad imprese che non hanno necessità di funzionalità 
      specifiche.
      \item \textbf{VisionENERGY}, gestionale con specifiche funzionalità pensate per le aziende che lavorano nel settore petrolifero, 
      come la possibilità di gestire la \textbf{vendita di carburante}, \textbf{manutenzione valvole}, ecc.;
      \item \textbf{VisionBLUE}, gestionale con specifiche funzionalità pensate per le aziende che lavorano nel settore ittico, 
      come la possibilità di gestire \textbf{lotti, prodotti e imballaggi};
      \item \textbf{VisionASSISTANCE}, gestionale con specifiche funzionalità pensate per le aziende specializzate nell'assistenza post-vendita, 
      come la possibilità di gestire \textbf{richieste di assistenza, contratti} e \textbf{assegnare gli ordini di intervento} ai 
      singoli tecnici;
      \item \textbf{VisionFRESH}, gestionale con specifiche funzionalità pensate per le aziende che lavorano nel settore ortofrutticolo, 
      come la possibilità di gestire \textbf{movimentazione merce, inserimento pesate, interfacciamento con bilance elettroniche}, ecc.;
      \item \textbf{VisionANTINCENDI}, gestionale con specifiche funzionalità pensate per le aziende che lavorano nel settore antincendi e antinfortunistica, 
      come la possibilità di gestire \textbf{chiamate ed interventi straordinari, buoni di manutenzione e geolocalizzare gli interventi};
      \item \textbf{VisionTRASPORTI}, gestionale con specifiche funzionalità pensate per le aziende che lavorano nel settore trasporti, 
      come la possibilità di gestire \textbf{listini, anagrafiche, dotazioni, manutenzione, pianificazione viaggi}, ecc.
\end{itemize}
Nella linea di prodotti MoviDAT invece troviamo una gamma di applicazioni sviluppate per i principali sistemi operativi per dispositivi 
\textit{mobile}: Android e iOS. Queste applicazioni sono state sviluppate per integrarsi direttamente con i gestionali 
della linea Vision e permettono di semplificare il lavoro di dipendenti che operano in mobilità e non hanno a disposizione un 
\textit{computer} con cui lavorare durante le trasferte (ed anche se ce lo avessero il suo utilizzo risulterebbe scomodo).\\
In questa linea dunque troviamo:
\begin{itemize}
    \item \textbf{MoviDOC} è un \gls{webapp} (ovvero un app a cui è possibile accedere direttamente da \textit{browser} senza 
          necessità di installarla sul dispositivo) che consente la gestione e condivisone dei documenti;
    \item \textbf{Handy} è un app per palmare che integrata a VisionENTERPRISE supporta la movimentazione della merce del magazzino o del punto vendita;
    \item \textbf{MoviSELL} è un app sviluppata per tablet iOS dedicata agli agenti aziendali, permette di: \textbf{visualizzare i 
          clienti su una mappa}, avere visibilità dello \textbf{stato contabile} e \textbf{inserire ordini clienti direttamente nel 
          ciclo attivo dell'azienda}.
    \item \textbf{MoviREP} è un app sviluppata per tablet iOS per la gestione digitalizzata dei rapportini da parte di operatori addetti alla manutenzione o 
          all'assistenza post vendita. 
    \item \textbf{MoviALERT} è una \gls{webapp} che permette di inviare mail di notifica automatiche all'avvenire di 
          specifici eventi nel gestionale;
    \item \textbf{MoviCHECK} è un \gls{webapp}, scaricabile anche su dispositivi Android e iOS per 
          consultare i dati di business in mobilità;
    \item \textbf{MoviEXPENSE} è un app per Android e iOS, per la registrazione automatica delle note 
          spese;
    \item \textbf{MoviCHECKIN} è una \gls{webapp} per la registrazione dei visitatori in azienda;
    \item \textbf{MoviORDER} applicazione per \textit{smartphone e tablet} iOS e Android che l’azienda può fornire ai propri clienti per l’invio di ordini e 
          richieste di approvvigionamento.
\end{itemize}

Nel caso in cui un'azienda richieda funzionalità specifiche per uno dei \textit{software} sopra elencati, {\company} offre la possibilità di creare una versione 
modificata dei propri prodotti. Per evitare di avere troppe variazioni della stesso prodotto il codice delle personalizzazioni (così vengono chiamate le funzioni in 
più richieste dal cliente) vengono inserite direttamente nel codice del \textit{software} principale, e "attivate" da specifici parametri controllati all'avvio del 
sistema. Nel caso di MoviORDER, che ho avuto la possibilità di esaminare per questo progetto, a seconda del valore del campo \textit{Company} ottenuto a seguito dell'
autenticazione del cliente venivano apportate alcune variazioni grafiche (loghi, tema). Questo si può ottenere grazie ad un'attenta progettazione e appropriate scelte 
architetturali.

 