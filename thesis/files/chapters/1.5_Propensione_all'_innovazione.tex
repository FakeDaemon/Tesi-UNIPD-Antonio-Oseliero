\section{Propensione all'innovazione}
{\company} non dispone di un ufficio specificamente dedicato alla ricerca e sviluppo, ma questo non significa che non 
vengano effettuati aggiornamenti costanti delle tecnologie e degli strumenti utilizzati. Ad esempio, l'azienda ha 
in programma di migrare i propri sistemi \gls{erp}, attualmente scritti in FoxPro (un linguaggio il cui supporto da 
parte di Microsoft è terminato nel 2015), verso tecnologie più moderne. Questo progetto, data la grandezza e 
complessità dei software coinvolti, richiederà anni per essere completato.\\
Inoltre, l'attività di stage rappresenta un'opportunità per l'azienda di innovare. Durante il mio tirocinio, 
ho osservato un apprezzamento particolare per l'indipendenza degli stagisti nel cercare e implementare 
soluzioni o tecnologie originali. Per ulteriori dettagli sul rapporto dell'azienda con gli stage, consultare il 
capitolo TODO 2.1.
