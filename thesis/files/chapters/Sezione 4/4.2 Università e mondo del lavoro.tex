\section{Università e mondo del lavoro}
Come ultimo argomento di questa tesi, vorrei offrire un parere riguardo la preparazione offerta 
dal {\myFaculty} del {\myUni}.\\
Comincio dicendo che reputo prezioso questo tempo passato in Università, che mi ha fatto crescere 
come persona e mi ha permesso di incontrare molte persone che mi hanno insegnato moltissimo.\\
Per quanto riguarda la preparazione offerta dal {\myFaculty}, ritengo che fornisca una base 
teorica robusta per affrontare il mondo del lavoro, pur mantenendosi abbastanza ampia e diversificata da permettere 
a chi desidera continuare il percorso di studi di individuare l'ambito in cui specializzarsi.\\
Per quanto riguarda la mia esperienza di \textit{stage} in particolare, ho trovato estremamente utili i corsi di 
Algoritmi e Strutture Dati, Basi di Dati, Tecnologie Web e Ingegneria del \textit{Software}, i quali mi 
hanno permesso di affrontare le sfide tecniche incontrate in azienda.\\
Tuttavia, l'unica critica che mi sento di muovere al percorso di studi riguarda la mancanza di un corso obbligatorio 
dedicato all'uso di Git e GitHub. Pur esistendo Metodologie e Tecnologie per lo Sviluppo 
\textit{Softare} che tratta questo argomento (e altri come gli strumenti per la 
\textit{Continuous Integration e Delivery}, metodologie Agile, \textit{best practice}, ecc.) rimane 
comunque un corso opzionale e che arriva troppo tardi nel corso di studi.\\
Se è vero che lo scopo dell'università non è quello di focalizzarsi su specifiche tecnologie ma 
fornire gli strumenti per imparare ad usare tutte quelle che incontreremo in futuro, è 
anche vero che Git è uno strumento imprescindibile per gran parte degli sviluppatori nel mondo 
in quanto fondamenta dello sviluppo collaborativo.