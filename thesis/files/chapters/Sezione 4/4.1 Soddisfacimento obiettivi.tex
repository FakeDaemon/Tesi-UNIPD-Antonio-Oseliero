\section{Soddisfacimento obiettivi}
\subsection{Requisiti e obiettivi aziendali}
Come ho riportato nel capitolo \ref{chap:strategia}, {\company} richiede agli studenti di presentare il proprio 
lavoro in un \textit{meeting} al termine dello \textit{stage}. In questa occasione, ho presentato al 
\textit{tutor} aziendale e agli sviluppatori che lavorano su {\movi} il lavoro svolto durante il tirocinio.\\
Durante questa riunione di circa quaranta minuti ho avuto l'opportunità di mostrare l'\textit{app} in funzione 
e di illustrare la logica di funzionamento del modulo in maniera dettagliata. 
Inoltre, ho potuto mostrare il lavoro svolto sull'implementazione degli \textit{unit test} e discutere dei 
benefici che possono portare al \textit{software}.\\
Tutti i presenti sono rimasti molto soddisfatti del mio lavoro e hanno ascoltato con attenzione la presentazione, 
in quanto l'implementazione del modulo agenti è una delle \textit{feature} previste per la nuova versione in sviluppo 
dell'\textit{app} una volta completata (io ho lavorato con la versione 3.1, mentre il \textit{team} di {\company} stava 
lavorando alla versione 4.0). La riunione si è conclusa con la conferma da parte del \textit{tutor} del pieno 
soddisfacimento di tutti i requisiti concordati, riportati dalla tabella \ref{tab:requisiti}.\\
Anche gli obiettivi aziendali, elencati dalla tabella \ref{tab:obiettivi}, sono stati tutti raggiunti con 
successo. Il soddisfacimento dei primi cinque obiettivi ("Modifica del sistema di autenticazione", 
"Aggiornamento delle interfacce per adeguamento al nuovo modulo", "Sviluppo del modulo agenti", "Stesura 
della documentazione tecnica e operativa" e "Ottimizzazione per \textit{tablet}") è dimostrato dal soddisfacimento 
dei requisiti che ho identificato durante l'attività di analisi.\\
Per quanto riguarda l'obiettivo "\textit{Testing} delle nuove funzionalità", 
non mi sono limitato a descrivere una serie di \textit{test} manuali nel documento di specifica tecnica. Ho infatti
superato le aspettative e le richieste del \textit{tutor} implementando una \textit{suite} di \textit{unit test} per 
il \textit{testing} automatizzato del codice, come ho descritto nel capitolo \ref{chap:verifica}.

\subsection{Obiettivi personali e conoscenze acquisite}
Nella tabella \ref{tab:obiettivi personali} ho riportato gli obiettivi formativi che mi ero prefissato in occasione 
dell'inizio dell'attività di tirocinio. Per me lo \textit{stage} si è rivelato un'occasione importante non solo 
per fare esperienza di sviluppo, ma anche per entrare a contatto con dei professionisti e affiancarmi a loro 
per lo sviluppo del progetto.\\
Per quanto riguarda il primo obiettivo, "Raggiungere una buona conoscenza delle principali tecnologie utilizzate 
(React Native e .NET)", posso affermare di aver raggiunto quanto prefissato. All'inizio della mia esperienza di \textit{stage},
possedevo una conoscenza solo basilare di React, e nessuna familiarità con .NET. Per questo motivo ho dedicato molto tempo 
allo studio approfondito delle guide ufficiali e all'analisi dettagliata del codice creato da {\company}. Inoltre ho 
avuto modo rafforzare le conoscenze acquisite applicandole in maniera concreta in un processo di apprendimento 
continuo.\\
Posso ritenere soddisfatto anche il secondo obiettivo che mi ero prefissato, "Analizzare e comprendere l’architettura 
\textit{software} di {\movi}", che ho descritto nel capitolo \ref{chap:progettazione}. La comprensione dell'architettura 
dell'\textit{app} mi ha permesso di ampliare il codice esistente senza intaccare le funzionalità pre esistenti, 
oltre che a poter creare un codice di qualità: mantenibile, testabile e modulare.\\
Il contatto diretto con il \textit{team} di sviluppatori mi ha aiutato ad apprendere il loro metodo di lavoro, 
ho assistito ai \textit{meeting} e ho visto come le diverse figure aziendali 
interagiscono tra loro. In questo modo ho potuto apprendere le metodologie di sviluppo di {\company} e 
imparare come utilizzano gli strumenti per gestire il \textit{framework} Scrum. In questo modo ho potuto 
realizzare il progetto con la sicurezza di creare un prodotto che rispondesse alle necessità esposte dal \textit{tutor}.
Ho descritto il processo di sviluppo nel capitolo \ref{chap:sviluppo} e ritengo quindi soddisfatto l'obiettivo 
"Apprendere ed implementare le metodologie di sviluppo aziendali".\\
Lo stretto contatto con i dipendenti dell'azienda mi ha aiutato a superare i punti più difficili dello sviluppo del 
modulo agenti. Si sono sempre dimostrati tutti disponibili a dare una mano e condividevano con me un opinione sulla 
qualità del lavoro e come poterlo migliorare. Durante la realizzazione di una componente chiave del modulo, la cui 
implementazione si stava rivelando particolarmente complessa, ho potuto fare una seduta di \textit{brainstorming} con 
uno sviluppatore di {\movi} e il \textit{project manager}. Abbiamo discusso a lungo, ma siamo riusciti a trovare 
una soluzione che ben si integrava con il resto del sistema, portando alla realizzazione di \texttt{AdditionalLogin} 
e il conseguente meccanismo di gestione dei due \textit{token}. Per questo posso ritenermi soddisfatto delle 
interazioni avute con il personale di {\company} e delle soluzioni applicate per la realizzazione del modulo, 
affermando di aver soddisfatto gli obiettivi "Collaborare con il \textit{team} di {\company} per 
la realizzazione del progetto" e "Migliorare nel \textit{problem solving} trovando soluzioni efficienti ai problemi 
incontrati durante lo sviluppo".\\
In conclusione ritengo di aver raggiunto tutti gli obiettivi che mi ero preposto, e di aver sfruttato appieno l'esperienza 
di tirocinio per accrescere la mia esperienza nell'ambito dello sviluppo \textit{software}.