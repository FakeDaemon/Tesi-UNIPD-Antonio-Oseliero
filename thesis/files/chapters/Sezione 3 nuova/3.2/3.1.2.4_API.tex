\subsubsection{\textit{Presentation layer} e API \textit{Controller}}
Il \textit{Presentation layer} rappresenta lo strato più esterno dell'architettura API di {\movi}, fungendo da interfaccia 
tra il sistema \textit{back-end} e il \textit{front-end}. Questo livello è implementato principalmente attraverso gli 
\textit{API Controller}, che sono responsabili della gestione delle richieste HTTP in ingresso e della formattazione 
delle risposte.\\
Questi \textit{controller} agiscono come punto di ingresso per le richieste \textit{client}, orchestrando il flusso 
di dati e le operazioni tra il \textit{client} e il \textit{Business Logic layer}.\\
Le principali responsabilità degli \textit{API Controller} includono:
\begin{itemize}
    \item \textbf{Gestione delle richieste}: Ricevono e interpretano le richieste HTTP in arrivo.
    \item \textbf{\textit{Routing}}: Indirizzano le richieste alle appropriate funzioni del \textit{Business Logic layer}.
    \item \textbf{Formattazione risposte}: Preparano e inviano risposte HTTP appropriate utilizzando i DTO.
\end{itemize}
Gli \textit{API Controller} interagiscono direttamente con il \textit{Business Logic layer}, in particolare con i 
servizi che vengono richiamati direttamente nel corpo del \textit{controller}.\\
Tutte le \gls{api} a parte quella per il \textit{login} non possono essere interrogate da un utente non autenticato 
(ovvero che non ha completato la procedura di \textit{login} e che non possiede un \textit{token} valido), in modo 
da concederne l'utilizzo solo a gli utenti di {\movi}.\\
All'interno della cartella dove sono definiti i \textit{controller} troviamo inoltre:
\begin{itemize}
    \item \textbf{\texttt{Program.cs}}: questo file è il punto di ingresso dell'applicazione, dove viene configurato e avviato il 
           \textit{server web} che ospita le \gls{api};
    \item \textbf{Configurazione di Swagger}: ovvero lo strumento usato per il \textit{testing} delle \gls{api};
    \item \textbf{Definizione degli \textit{endpoint}}: ovvero gli indirizzi di connessione ai vari \textit{database} 
          e l'indirizzo in cui le \gls{api} vengono esposte.
\end{itemize}
In conclusione, il \textit{Presentation layer} e gli \textit{API Controller} in {\movi} fungono da interfaccia 
tra il mondo esterno e la logica interna dell'applicazione. Attraverso una progettazione attenta e l'uso di DTO, questo 
\textit{layer} garantisce una comunicazione efficiente, sicura e flessibile con il \textit{front-end}, mantenendo al 
contempo una chiara separazione delle responsabilità all'interno dell'architettura complessiva del sistema.