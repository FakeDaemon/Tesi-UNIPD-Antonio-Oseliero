\section{Codifica}

Una volta compresa l'architettura \textit{software} dell'\textit{app} ho iniziato l'attività di codifica, dove ho tradotto 
i requisiti riportati dalla tabella \ref{tab:requisiti} in codice.\\
Questa attività ha richiesto una particolare attenzione all'integrazione del nuovo codice con quello esistente, 
in modo da non modificare il comportamento del \textit{software} per i clienti. Il modulo agenti mi ha richiesto 
l'estensione o la modifica di quasi ogni componente dell'architettura precedentemente descritta e mi ha permesso 
di esplorare il funzionamento dell'applicazione nella sua interezza.

\subsection{\textit{Database} e \textit{Data Access layer}}
Sono partito modificando la struttura del \textit{database} aggiungendo una colonna \texttt{IdUser} alla tabella 
\texttt{User} del \textit{common database}. Utilizzando le funzionalità di Entity Framework mi e bastato modificare 
il modello dei dati e lanciare i comandi per rendere effettive le modifiche anche nel \textit{database}. Con questa 
nuova colonna ora posso identificare i clienti dal fatto che la colonna \texttt{BPCode} risulta nulla e la colonna 
\texttt{IdUser} non nulla, viceversa i clienti.\\
La seconda modifica è stata l'importazione della tabella \texttt{Bp} del \textit{company database} che contiene i 
dati inerenti ai clienti aziendali. Il \textit{company database} è stato creato dall'\textit{app} MoviSELL, e 
{\movi} ne utilizza qualche tabella necessaria per svolgere le sue funzioni. Per questo motivo per integrare 
\texttt{Bp} nel modello di {\movi} ho lavorato al contrario utilizzando il comando \texttt{scaffold} di 
Entity Framework che permette di integrare una tabella già presente nel \textit{database} al modello.\\

\subsection{API e \textit{Business Logic layer}}
Per permettere il funzionamento del modulo agenti ho dovuto modificare una funzione del \textit{Business layer} 
e creare due nuove \gls{api}.\\
La funzione \texttt{Login}, richiamata dall'omonimo \gls{api} \textit{Controller}, è la funzione che si 
occupa di autenticare l'utente all'interno dell'\textit{app}. Le modifiche che ho apportato riguardano 
una serie di controlli aggiuntivi per permettere l'autenticazione degli agenti. Ho inoltre modificato le 
informazioni contenute nel \textit{token} restituito a gli agenti per ottimizzare la funzione 
\texttt{GetAgentCustomers} che verrà discussa in questo capitolo.\\
Una delle due \gls{api} che ho creato per il modulo agenti è \texttt{AdditionalLogin}, richiamata dall'omonimo 
\gls{api} \textit{Controller}. Questa funzione, definita nello stesso servizio di \texttt{Login}, permette 
di effettuare una seconda \textit{login} nell'\textit{app} saltando i controlli sulla \textit{password} e 
restituendo un nuovo \textit{token}. \texttt{AdditionalLogin} è fondamentale per permettere all'agente di 
autenticarsi nell'\textit{app} come il cliente selezionato grazie al sistema di gestione del doppio \textit{token} 
che ho implementato nel \textit{front-end}. La sicurezza di questa \gls{api} è assicurata dal fatto che 
può essere richiamata solo da un utente autenticato, e necessita quindi che l'utente che richiama l'\gls{api} 
sia un utente {\movi}.\\
L'ultima funzione che ho creato è \texttt{GetAgentCustomers}, richiamata dall'omonimo \gls{api} \textit{Controller} 
e sviluppata all'interno del nuovo servizio \texttt{BpService}. Questa funzione recupera tutte le informazioni 
inerenti ai clienti dell'agente e ritorna una lista di \texttt{BpCustomerDto}, un nuovo DTO che contiene una serie di 
informazioni ricavate sul cliente.

\subsection{\textit{Front-end}}

\begin{figure}[H]
    \centering
    \subfloat[\texttt{Homepage Agenti} definitiva di {\movi}]{
        \includegraphics[width=0.3\textwidth]{img/agenthomepage.jpg}
        \label{fig:MVOR homepage}
    }
    \hfill
    \subfloat[\texttt{Homepage} definitiva di {\movi}]{
        \includegraphics[width=0.3\textwidth]{img/homepage.jpg}
        \label{fig:MVOR agent homepage}
    }
    \caption{Versione finale delle \textit{homepages} di {\movi}}
    \label{fig:MVOR homepages}
\end{figure}

Lo sviluppo del modulo ha richiesto diverse aggiunte e modifiche anche alla parte di \textit{front-end}.\\
Per quanto riguarda l'interfaccia ho sviluppato la \textit{view} \texttt{Homepage Agenti} e tutti i 
\textit{component} utilizzati dalla pagina. Questa \textit{view} è il luogo in cui viene portato l'utente 
dopo aver completato l'autenticazione con successo ed esser stato riconosciuto come agente aziendale. Qui 
l'agente può accede alla pagina \texttt{Impostazioni} e \texttt{Account}, oppure consultare la lista dei 
propri clienti. Una volta selezionato un cliente dalla lista, l'agente viene portato nella \texttt{Homepage} 
dove può accedere a tutte le funzionalità dell'\textit{app} e operare come il cliente selezionato.\\
Ovviamente si è visto necessaria la modifica della \texttt{Homepage}, aggiungendo un pulsante che permettesse 
all'agente di tornare alla selezione del cliente (componente nascosto ai clienti) e la rimozione della possibilità 
per gli agenti di modificare la propria schermata di avvio.\\
Entrambe le viste descritte sono mostrate dall'immagine \ref{fig:MVOR homepages}.\\
Ovviamente per integrare queste nuove \textit{view} nel sistema ho dovuto modificare lo \texttt{Store} Redux 
e lo \textit{stack} di navigazione di \texttt{Navigation} in modo da rendere visualizzabili le modifiche 
solo ai clienti e per permettere una corretta visualizzazione della pagina e i sui componenti sia nel caso 
l'agente effettui l'operazione di \textit{login}, sia nel caso in cui l'agente abbia salvato le proprie 
credenziali in modo da rendere più rapido l'accesso al sistema.\\
La seconda modifica importante al \textit{front-end} e la possibilità per il sistema di gestire due \textit{token}. 
Il secondo \textit{token} viene restituito da \texttt{AdditionalLogin} in seguito alla selezione da parte 
dell'agente di un cliente dalla lista. Questo \textit{token} è fondamentale per richiamare alcune \gls{api} all'interno 
delle pagine dell'\textit{app}. Per questo, controllando una variabile booleana nello stato, la funzione \texttt{getToken} 
decide quale dei due \textit{token} ritornare permettendo la corretta integrazione del modulo al sistema.\\
Il lavoro inerente il lato estetico e di usabilità dell'\textit{app} ha compreso:
\begin{itemize}
    \item \textbf{Ottimizzazione per \textit{tablet}}: ho adeguato il CSS della \textit{view} \texttt{Homepage Agenti} in 
          modo da renderla utilizzabile anche da \textit{tablet};
    \item \textbf{Modifiche generali}: leggere modifiche allo stile di \textit{component} e \textit{view} pre esistenti;
    \item \textbf{Adeguamento di \texttt{Homepage Agenti} al tema selezionato}: possibilità di selezionare il tema chiaro 
          o scuro anche per \texttt{Homepage Agenti};
    \item \textbf{Adeguamento del modulo agenti alla lingua di sistema}: possibilità di vedere le nuove frasi 
          introdotte nell'\textit{app} con il modulo agenti in italiano e inglese.
\end{itemize}