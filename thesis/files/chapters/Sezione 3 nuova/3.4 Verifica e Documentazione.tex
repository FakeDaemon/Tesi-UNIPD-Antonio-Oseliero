\section{Verifica e documentazione}
\subsection{\textit{Testing}}
Un'attività fondamentale del ciclo di sviluppo è il \textit{testing} delle modifiche apportate al sistema.\\
{\company} non utilizza strumenti per la verifica automatica del codice, e ho quindi testato manualmente tutte 
le novità introdotte e la corretta integrazione del modulo agenti.\\
Tuttavia, per mia iniziativa e in accordo con il \textit{tutor} aziendale, ho deciso di implementare una \textit{suite} di 
\textit{test} automatici per alcune funzioni della \textit{Business Logic} del \textit{back-end} utilizzando il \textit{framework} 
Moq e xUnit, un \textit{tool} per creare \textit{unit test} in .NET.\\
Moq è uno strumento per la creazione di \textit{mock} e \textit{stub} in \textit{unit test} 
e \textit{integration test}. Permette di simulare il comportamento di oggetti e componenti esterni al codice che si sta 
testando, sostituendoli con implementazioni fittizie ma controllabili. Nello specifico, Moq consente di:
\begin{itemize}
    \item \textbf{Creare \textit{mock} di interfacce e classi concrete}, in modo da poter testare il codice in modo isolato 
          senza dipendere dalle reali implementazioni.
    \item \textbf{Definire aspettative sui metodi testati}, verificando che il codice che si sta testando 
          interagisca correttamente con i \textit{mock} forniti.
\end{itemize}
Con l'utilizzo di Moq ho potuto simulare le chiamate a \textit{database} e ad altre funzioni, restituendo oggetti creati \textit{ad hoc}
per testare diverse casistiche e verificare il corretto comportamento delle funzioni testate, indipendentemente 
dall'integrazione con il resto del sistema.\\
La prima difficoltà che ho incontrato in questa fase è stata la creazione di oggetti \textit{mock} particolarmente 
complessi. A volte infatti i servizi richiedevano l'utilizzo di oggetti con molti attributi spesso non banali.\\
Inoltre, la versione gratuita di Moq non mi permetteva di simulare il comportamento di metodi non pubblici 
e virtuali, costringendomi a modificare la firma di alcune funzioni.\\
Sebbene questo non rappresenti l'approccio ideale per l'implementazione di \textit{test} unitari, ho comunque deciso di 
cambiare la firma delle funzioni. La mia motivazione principale era quella di dimostrare l'efficacia dei \textit{test} 
automatici nell'individuare e prevenire alcuni problemi dati dalla continua evoluzione del codice, come 
l'introduzione di regressione nelle funzionalità del sistema. Si parla di regressione quando una funzionalità del 
sistema presenta un comportamento differente e non previsto con l'introduzione di novità nel codice.\\
Ho anche provato ad implementare una serie di \textit{test} automatici per le funzioni del \textit{front-end}, ma per mancanza 
di tempo ho preferito dare priorità alla stesura della documentazione, e ho quindi rinunciato.\\

\subsection{Documentazione}
Un attività fondamentale dello sviluppo è quello della stesura della documentazione.\\
La documentazione è una componente cruciale dello sviluppo \textit{software}, non solo perché facilita la manutenzione 
e l'evoluzione del \textit{software}, ma anche perché aiuta i nuovi sviluppatori a comprendere rapidamente il sistema 
e supporta gli utenti finali nell'utilizzo efficace del prodotto.\\
Come richiesto dal \textit{tutor} aziendale, ho prodotto due documenti inerenti il modulo agenti: un documento di 
specifica tecnica e un manuale utente.\\
Il manuale utente illustra in maniera semplice e precisa le funzionalità offerte dalla nuova versione di {\movi}, 
senza utilizzare linguaggio tecnico. Per scrivere questo documento ho utilizzato il \textit{software} Figma, come 
riportato nel capitolo \ref{chap:vincoli tec}, in modo da avere maggiori strumenti per curare l'estetica del documento. 
Abbondano anche le immagini, che forniscono un supporto visivo al testo e aiutano l'utente nell'utilizzo delle nuove 
funzionalità.\\
Viceversa il documento di specifica tecnica risulta essenziale dal punto di vista estetico e utilizza un linguaggio 
tecnico e settoriale descrivendo in maniera precisa tutte le modifiche introdotte e la logica di funzionamento 
del modulo agenti. Questo documento è destinato a sviluppatori e addetti ai lavori, in modo che possa servire in 
futuro a guidare l'implementazione del modulo nel prodotto reale.\\
Nella specifica tecnica, nell'apposita sezione, ho descritto una serie di \textit{test} manuali, sia per l'interfaccia 
che per le \gls{api}, in modo da dimostrare la corretta integrazione del nuovo modulo con il sistema esistente.