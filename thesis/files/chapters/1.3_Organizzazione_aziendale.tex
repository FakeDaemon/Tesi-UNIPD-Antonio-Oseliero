\section{Organizzazione aziendale}
{\company} è suddivisa in tre aree di competenza distinte che vedono ruoli e responsabilità distinte tra loro.\\
Reparto Assistenza: qui troviamo i consulenti tecnici gestionali, che hanno il compito di aiutare l'azienda nell'implementazione dei nuovi gestionali 
e nella gestione del cambiamento assicurandosi che il personale aziendale sia formato sull'uso delle nuove tecnologie. Ogni consulente è responsabile di uno 
o più \textit{software} di cui hanno un ampia conoscenza operativa. Per questo svolgono anche la mansione di offrire assistenza ai clienti ed aiutandoli 
nella risoluzione dei problemi, eventualmente segnalandoli al reparto sviluppo.\\
Area Amministrazione: Commerciale e Marketing: in questa area troviamo varie competenze: dall'addetto al marketing alle risorse umane, 
la contabilità, la segreteria generale e commerciale, gli addetti alle vendite e il responsabile d'impatto (responsabile della valutazione, 
pianificazione e promozione delle misure di \gls{csr} ovvero di un agenzia benefit, ovvero di tutte le misure attuate dall'azienda in ambito sociale 
e transizione ecologica).\\
Reparto Sviluppo \textit{Software}: in quest'area lavorano gli sviluppatori che coordinandosi con l'analista, che ha il compito di raccogliere 
i requisiti, il \textit{project manager}, che ha il compito di pianificare e coordinare le attività, e il direttore dello sviluppo, che il compito di 
monitorare e guidare il team di sviluppo, creano il codice dei prodotti commercializzati dall'azienda, li mantengono e li espandono.\\
Il tutor aziendale che mi ha seguito durante l'esperienza di stage, Turra Francesco, ricopre il ruolo di analista, amministratore, \textit{product manager}
 e referente commerciale rete indiretta.
