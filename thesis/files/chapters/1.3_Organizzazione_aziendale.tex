\section{Organizzazione aziendale}
{\company} è strutturata in tre aree di competenza, ognuna con ruoli e responsabilità specifiche.\\
\textbf{Reparto Assistenza}: qui operano i consulenti tecnici gestionali, il cui compito è assistere l'azienda nell'implementazione dei nuovi gestionali 
e nella gestione del cambiamento assicurandosi che il personale aziendale sia formato sull'uso delle nuove tecnologie. Ogni consulente è responsabile di uno 
o più \textit{software} di cui hanno un ampia conoscenza operativa. Inoltre, forniscono assistenza ai clienti, aiutandoli 
nella risoluzione dei problemi e, se necessario, segnalando le problematiche al reparto sviluppo.\\
\textbf{Area Amministrazione Commerciale e \textit{Marketing}}: In quest'area si trovano diverse competenze, tra cui \textbf{il responsabile \textit{marketing}, risorse umane, contabilità, 
segreteria generale e commerciale, responsabile vendite e il responsabile d'impatto}. Quest'ultimo si occupa della valutazione, pianificazione e promozione delle misure 
di responsabilità sociale d'impresa (\gls{csr}), ovvero di tutte le iniziative attuate dall'azienda in ambito sociale e di transizione ecologica.\\
\textbf{Reparto Sviluppo \textit{Software}}: In quest'area lavorano gli \textbf{sviluppatori}, coordinati dall'\textbf{analista}, che raccoglie i requisiti, dal 
\textbf{\textit{project manager}}, che pianifica e coordina le attività, e dal \textbf{direttore dello sviluppo}, che guida e monitora il \textit{team}. Insieme, creano, mantengono 
ed espandono il codice dei prodotti commercializzati dall'azienda.\\
Il tutor aziendale che mi ha seguito durante l'esperienza di stage, Turra Francesco, ricopre il ruolo di \textbf{analista, amministratore, \textit{product manager}
 e referente commerciale per la rete indiretta}.
