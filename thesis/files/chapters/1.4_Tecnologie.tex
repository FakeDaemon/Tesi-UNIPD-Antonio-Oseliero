\section{Tecnologie}\label{chap:Tecnologie}
L'azienda utilizza diversi strumenti sia per lo sviluppo, che per lo svolgimento dei normali processi aziendali.
\begin{itemize}
    \item \textbf{Portatili}: ad ogni impiegato viene messo a disposizione un portatile con Windows 10 o 11, il sistema 
          operativo di Microsoft o all'occorrenza un Mac con macOS, il portatile sviluppato da Apple con il suo sistema operativo proprietario;
    \item \textbf{Dispositivi \textit{mobile}}: all'interno dell'azienda troviamo molti dispositivi \textit{mobile} con diversi sistemi operativi e dimensioni 
          dello schermo, usati per il \textit{testing} delle applicazioni Android e iOS;
    \item \textbf{Microsoft Office 365}: servizio in abbonamento di Microsoft che include diversi software come Word e PowerPoint;
    \item \textbf{Zimbra}: sistema di posta elettronica utilizzato dall'azienda, durante il mio stage è stato cambiato in favore di un'integrazione di Zimbra 
          con Outlook;
    \item \textbf{Bitbucket}: strumento per la gestione della versione Git basato sul \textit{web}, che consente di creare 
          \textit{repository} pubbliche o private per caricare il proprio codice e gestirlo in modo collaborativo con il proprio 
          \textit{team}. Qui gli sviluppatori caricano il loro codice suddiviso in \textit{repository} per ogni progetto. Ogni \textit{repository} vede diversi 
          \textit{branch} attivi: \textbf{\textit{main}} che contiene l'ultima versione rilasciata al pubblico del software, \textbf{\textit{develop}} ovvero il 
          \textit{branch} di lavoro dove nascono e confluiscono tutti i \textit{feature branch} prima di effettuare il rilascio in \textit{main}, i 
          \textbf{\textit{feature branch}} che viene creato dal programmatore per sviluppare una specifica funzione del programma che sarà, una volta terminata e
          testata, aggiunta in \textit{develop};
    \item \textbf{Jira}: \textit{suite} di \textit{software} proprietari per il tracciamento delle segnalazioni sviluppato
           da Atlassian, che consente il \textit{bug tracking} e la gestione dei progetti. Questo software è particolarmente utile per pianificare i vari 
           compiti da svolgere nello \textit{sprint} e assegnarli ai vari componenti del \textit{team} di sviluppatori. Questi compiti sono chiamati \textit{ticket}
           e possono essere di diverso tipo: \textbf{\textit{epic}} che rappresentano grosse porzioni di lavoro e sono quindi usate come raccolte di \textit{ticket},
           \textbf{\textit{task}} il singolo compito che deve essere completato e \textbf{\textit{bug}} che rappresenta una problematica da risolvere.
           I \textbf{\textit{bug}} possono essere avere diverse origini: gli sviluppatori stessi nel caso in cui si accorgano di un difetto di programmazione o da i 
           clienti che telefonando all'assistenza riportano il problema, quindi il tecnico riporterà la problematica al \textit{project manager} che creerà la \textit{task};
    \item \textbf{Confluence}: strumento che permette ai \textit{team} di creare, condividere e collaborare su documenti e contenuti in un ambiente 
          centralizzato e strutturato;
    \item \textbf{3CX}: centralino telefonico PBX (\textit{Private Branch Exchange}), ovvero una rete telefonica privata utilizzata all'interno di un'azienda 
          o organizzazione. Gli utenti del sistema telefonico PBX possono comunicare internamente ed esternamente, tramite il classico telefono fisso o 
          chat da \textit{smartphone}. Questo sistema permette anche di effettuare video chiamate e di scambiarsi messaggi all'interno di \textit{chat};
    \item \textbf{Visual Studio}: Si tratta di un ambiente di sviluppo integrato completo (\gls{ide}) che è possibile usare per scrivere, modificare, 
          eseguire il \textit{debug} e compilare codice. Visual Studio include \textbf{compilatori, strumenti di completamento del codice, 
          controllo del codice sorgente, estensioni} e molte altre funzionalità per migliorare ogni fase del processo di sviluppo \textit{software};
    \item \textbf{Visual Studio Code}: \textit{editor} di codice sorgente particolarmente leggero ed estensibile grazie ad una gamma di estensioni 
          che è possibile integrargli. È inoltre \textit{open source} e compatibile con una vasta gamma di sistemi operativi;
    \item \textbf{SQL Server Management Studio (SSMS)}: è un ambiente integrato per la configurazione, la gestione e l'amministrazione di tutti i componenti, 
          le istanze e i database all'interno di Microsoft SQL Server. SSMS include sia \textit{editor} di \textit{script} che strumenti grafici che lavorano con 
          oggetti e funzionalità del \textit{server}.
\end{itemize}
L'azienda utilizza una vasta gamma di linguaggi di programmazione, \textit{framework} e librerie per diversi motivi. Alcuni di questi includono l'acquisizione 
e l'adattamento di codice sorgente da altre aziende, il fatto che il codice sia stato scritto molti anni fa con tecnologie ormai obsolete, e la necessità di utilizzare 
linguaggi specifici per soddisfare esigenze particolari. Tuttavia, l'azienda si impegna a uniformare quanto più possibile i linguaggi e a ridurre il numero di tecnologie in 
uso, al fine di semplificare e rendere più efficiente la gestione delle risorse tecnologiche. Oltre a quelle che ho usato per il mio progetto (che verranno discusse 
più approfonditamente nel capitolo TODO 2.4.1) queste sono alcune delle tecnologie che l'azienda usa per lo sviluppo dei suoi prodotti.
\begin{itemize}
    \item \textbf{Angular}: \textit{framework} per lo sviluppo di \gls{webapp} basato su TypeScript, sviluppato e mantenuto da Google;
    \item \textbf{Librerie Dev Express}: Dev Express è un'azienda incentrata sulla creazione di librerie di componenti grafici di cui le più famose sono 
          Blazor e MAUI. Queste librerie sono supportate per lo sviluppo in React, Angular e Vue;
    \item \textbf{FoxPro}: un sistema di gestione di \textit{database} e un linguaggio di programmazione procedurale orientato agli oggetti. Originariamente sviluppato 
          da Fox Software e successivamente acquisito da Microsoft. 
\end{itemize}