\cleardoublepage
\phantomsection
\pdfbookmark{Compendio}{Compendio}
\begingroup
\let\clearpage\relax
\let\cleardoublepage\relax
\chapter*{Sommario}

Il presente documento descrive il lavoro svolto dallo studente {\myName} durante il suo \textit{stage} presso l’azienda {\companyLong}.\\
Lo scopo del tirocinio è stato studiare il codice dell’applicazione mobile {\movi} e sviluppare un modulo di autenticazione di agenti aziendali 
in un'applicazione pensata per clienti terzi. Più precisamente era richiesto che, dopo l’autenticazione, l’agente possa scegliere un cliente da una 
lista e operare all’interno dell’applicazione come il cliente selezionato senza la necessità di autenticarsi come tale.\\
Raggiungere questo obiettivo ha richiesto lo studio del codice e dell’architettura dell’applicazione: \textit{front-end, back-end} e base dati 
sottostante, e un certo insieme di tecnologie e strumenti tra i quali React Native e ASP.NET Core, Visual Studio, e Server Management Studio. 
Il progetto ha incluso la realizzazione delle \gls{api} e l’interfaccia grafica del modulo richiesto, insieme a una batteria di \textit{test} automatici per la 
\textit{Business Logic} di alcune di tali \gls{api}.\\\\

Ho segnalato tutte le parole non italiane in \textit{corsivo} all'interno del documento.\\
Utilizzo il carattere \texttt{monospaziato} per i nomi di tabelle e colonne del \textit{database}, 
classi, funzioni, \textit{component, view} o altre parti del codice.\\
Evidenzio le parole del glossario con una G a pedice, in corsivo e in blu (ad esempio \gls{csr}).\\
Uso il \textbf{grassetto} per enfatizzare la parola chiave di un punto elenco 
per migliorare la leggibilità e rendere immediatamente identificabili i concetti chiave.\\
Ecco come appare un elenco puntato:\\
"Elenco numeri primi:
\begin{itemize}
\item \textbf{1}
\item \textbf{2}
\item \textbf{3}
\end{itemize}
..."\\\\
Ho diviso il documento in 4 macro sezioni:\\
\textbf{Capitolo 1 - VisioneImpresa}: Descrivo l'azienda dove ho svolto il tirocinio, riportando brevemente clienti, 
prodotti, organizzazione aziendale, strumenti e tecnologie utilizzate, e la propensione all'innovazione.\\
\textbf{Capitolo 2 - Descrizione e pianificazione del progetto di \textit{stage}}: Presento il progetto assegnatomi dall'azienda, specificando 
obiettivi, vincoli tecnologici e temporali. Approfondisco inoltre il rapporto dell'azienda con gli \textit{stage} in generale e 
le motivazioni dietro la scelta di questo specifico progetto.\\
\textbf{Capitolo 3 - Sviluppo del modulo agenti}: Descrivo le attività del processo di sviluppo adottato per realizzare il progetto, 
dall'analisi dei requisiti al \textit{testing}.\\
\textbf{Capitolo 4 - Retrospettiva}: Offro un giudizio obiettivo sul raggiungimento degli obiettivi di \textit{stage}, 
personali e aziendali. Faccio inoltre un resoconto delle conoscenze acquisite con questa esperienza e una valutazione 
personale del percorso di studi universitario.

\endgroup
\vfill
