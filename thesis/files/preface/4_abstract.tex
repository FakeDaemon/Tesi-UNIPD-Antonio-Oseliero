\cleardoublepage
\phantomsection
\pdfbookmark{Compendio}{Compendio}
\begingroup
\let\clearpage\relax
\let\cleardoublepage\relax
\chapter*{Sommario}

Il presente documento descrive il lavoro svolto dallo studente {\myName} durante il suo \textit{stage} presso l’azienda {\companyLong}.\\
Lo scopo del tirocinio è stato studiare il codice dell’applicazione mobile {\movi} e sviluppare un modulo di autenticazione di agenti aziendali 
in un'applicazione pensata per clienti terzi. Più precisamente era richiesto che, dopo l’autenticazione, l’agente possa scegliere un cliente da una 
lista e operare all’interno dell’applicazione come il cliente selezionato senza la necessità di autenticarsi come tale.\\
Raggiungere questo obiettivo ha richiesto lo studio del codice e dell’architettura dell’applicazione: \textbf{\textit{front-end, back-end} e base dati 
sottostante}, e un certo insieme di tecnologie e strumenti tra i quali \textbf{React Native e ASP.NET Core, Visual Studio, e Server Management Studio}. 
Il progetto ha incluso la realizzazione delle API e l’interfaccia grafica del modulo richiesto, insieme a una batteria di test automatici per la 
\textit{Business Logic} di alcune di tali API.\\\\

Segnalerò tutte le parole non italiane in \textit{corsivo} all'interno del documento.\\
Utilizzerò il carattere \texttt{monospaziato} per i nomi di tabelle e colonne del \textit{database}, 
classi, funzioni, \textit{component, view} o altre parti del codice.\\
Evidenzierò le parole del glossario con una G a pedice, in corsivo e in blu (ad esempio \gls{csr}).\\
Userò il \textbf{grassetto} per enfatizzare la parola chiave di un punto elenco o un singolo elemento di una lista,
per migliorare la leggibilità e rendere immediatamente identificabili i concetti chiave.\\
Ecco come verrà visualizzata una lista:
"I numeri primi sono \textbf{1, 2, 3, 7}..."\\
Ed ecco come apparirà un elenco puntato:
"Elenco numeri primi:
\begin{itemize}
\item \textbf{1}
\item \textbf{2}
\item \textbf{3}
\end{itemize}
..."\\\\
Ho diviso il documento in 4 macro sezioni:\\
\textbf{Capitolo 1 - VisioneImpresa}: Descriverò l'azienda dove ho svolto il tirocinio, riportando brevemente clienti, 
prodotti, organizzazione aziendale, strumenti e tecnologie utilizzate, e la propensione all'innovazione.\\
\textbf{Capitolo 2 - Descrizione e pianificazione \textit{stage}}: Presenterò il progetto assegnatomi dall'azienda, specificando 
obiettivi, vincoli tecnologici e temporali. Approfondirò inoltre il rapporto dell'azienda con gli \textit{stage} in generale e 
le motivazioni dietro la scelta di questo specifico progetto.\\
\textbf{Capitolo 3 - \textit{Stage}}: Racconterò la mia esperienza di \textit{stage}.\\
\textbf{Capitolo 4 - Retrospettiva}: Offrirò un giudizio obiettivo sul raggiungimento degli obiettivi di \textit{stage}, 
personali e aziendali. Farò inoltre un resoconto delle conoscenze acquisite con questa esperienza e una valutazione 
personale del percorso di studi universitario.

\endgroup
\vfill
