\cleardoublepage
\phantomsection
\pdfbookmark{Compendio}{Compendio}
\begingroup
\let\clearpage\relax
\let\cleardoublepage\relax
\chapter*{Sommario}

Il presente documento descrive il lavoro svolto dallo studente {\myName} durante il suo stage presso l’azienda {\companyLong}.\\
Lo scopo del tirocinio è stato studiare il codice dell’applicazione mobile {\movi} e sviluppare un modulo di autenticazione di agenti aziendali 
in un'applicazione pensata per clienti terzi. Più precisamente era richiesto che, dopo l’autenticazione, l’agente possa scegliere un cliente da una 
lista e operare all’interno dell’applicazione come il cliente selezionato senza la necessità di autenticarsi come tale.\\
Raggiungere questo obiettivo ha richiesto lo studio del codice e dell’architettura dell’applicazione: \textbf{\textit{front-end, back-end} e base dati 
sottostante}, e un certo insieme di tecnologie e strumenti tra i quali \textbf{React Native e ASP.NET Core, Visual Studio, e Server Management Studio}. 
Il progetto ha incluso la realizzazione delle API e l’interfaccia grafica del modulo richiesto, insieme a una batteria di test automatici per la 
\textit{Business Logic} di alcune di tali API.\\\\

All'interno del documento tutte le parole inglesi verranno segnalate in \textit{corsivo}.\\
Le parole facenti parte del glossario saranno evidenziate tramite una G a pedice, il corsivo e la parola sarà di colore blu (esempio \gls{csr}).\\
Nel documento a volte verrà utilizzato il \textbf{grassetto} per enfatizzare la parola chiave di un punto di un elenco puntato o il singolo elemento di un elenco. 
Questo per migliorare la leggibilità del documento e rendere subito identificabili i concetti chiave di una lista.\\
Ad esempio una lista sarà cosi visualizzata:\\
"i numeri primi sono \textbf{1,2,3,7}...", \\
mentre un elenco puntato apparirà nel seguente modo:\\
"elenco numeri primi:
\begin{itemize}
    \item \textbf{1};\item \textbf{2};\item \textbf{3};
\end{itemize}
..."\\\\


Il documento è diviso in 4 macro sezioni:\\
\textbf{Capitolo 1 - VisioneImpresa}: descrizione dell'azienda in cui ho svolto il tirocinio riportando brevemente clienti, prodotti, organizzazione aziendali, strumenti e 
tecnologie utilizzate, propensione dell'azienda all'innovazione;\\
\textbf{Capitolo 2 - Descrizione e pianificazione stage}: qui si riporta il progetto assegnatomi dall'azienda riportando obiettivi, vincoli tecnologici e temporali, viene 
inoltre approfondito il rapporto dell'azienda con gli stage in generale e la motivazione dietro la scelta di questo specifico progetto;\\
\textbf{Capitolo 3 - Stage}: descrizione della mia esperienza di stage;\\
\textbf{Capitolo 4 - Retrospettiva}: giudizio obiettivo in merito il soddisfacimento degli obiettivi di stage, personali e aziendali, oltre 
ad un resoconto delle conoscenze acquisite con questa esperienza e una personale valutazione del percorso di studi universitario.


\endgroup
\vfill
