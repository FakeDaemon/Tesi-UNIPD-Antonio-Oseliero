% Load variables
\newcommand{\myUni}{Università degli Studi di Padova}
\newcommand{\myDepartment}{Dipartimento di Matematica ``Tullio Levi-Civita''}
\newcommand{\myFaculty}{Corso di Laurea in Informatica}
\newcommand{\myTitle}{MoviORDER, modulo agenti per gestione clienti}
\newcommand{\myDegree}{Tesi di Laurea Triennale}
\newcommand{\profTitle}{Prof.}
\newcommand{\myProf}{Vardanega Tullio}
\newcommand{\graduateTitle}{Laureando}
\newcommand{\myName}{Oseliero Antonio}
\newcommand{\myStudentID}{1226325}
\newcommand{\myAA}{2023-2024}
\newcommand{\myLocation}{Padova}
\newcommand{\myTime}{Settembre 2024}
% Acronyms
%\newacronym{api}{API}{Application Program Interface}
%\newacronym{sdk}{SDK}{Software Development Kit}

% Glossary
\newglossaryentry{api}{
    name={API},
    text={API},
    sort=api,
    description={ \textit{Application Programming Interface} (interfaccia di programmazione delle applicazioni) sono un insieme di definizioni 
    e protocolli con i quali vengono realizzati e integrati \textit{software applicativi}. Le API stabiliscono il contenuto e la forma dei dati 
    necessari per la chiamata e quelli restituiti in risposta.\\
    {[\textit{fonte riportata in sitografia}]}}
}

\newglossaryentry{csr}{
    name={CSR},
    text={CSR},
    sort=csr,
    description={   
                    Per Responsabilità Sociale delle Imprese (e delle organizzazioni) o secondo l'acronimo inglese CSR, 
                    \textit{Corporate Social Responsibility}, si intende l'integrazione su base volontaria, da parte delle imprese, 
                    delle preoccupazioni sociali e ambientali nelle loro operazioni interessate.\\
                    {[\textit{fonte riportata in sitografia}]}}
}

\newglossaryentry{erp}{
    name={ERP},
    text={ERP},
    sort=erp,
    description={
                    \textit{Enterprise Resource Planning}, è un tipo di sistema \textit{software} che aiuta le organizzazioni ad automatizzare e 
                    gestire i processi aziendali principali per ottenere le prestazioni ottimali. Il \textit{software} ERP coordina il flusso di dati 
                    tra i processi di un'azienda, fornendo un'unica fonte di informazioni e semplificando le operazioni nell'azienda. È in grado di 
                    collegare le attività finanziarie, della catena di approvvigionamento, delle operazioni, del commercio, dei \textit{report}, della 
                    produzione e delle risorse umane di un'azienda in una sola piattaforma.\\
                    {[\textit{fonte della definizione inglese riportata in sitografia}]}}
}

\newglossaryentry{ide}{
    name={IDE},
    text={IDE},
    sort=ide,
    description={
                    \textit{Integrated Development Environment} o ambiente di sviluppo integrato, è un \textit{software} progettato per la 
                    realizzazione di applicazioni che aggrega strumenti di sviluppo comuni in un'unica interfaccia utente grafica. In genere è costituito da: 
                    \textit{editor} del codice sorgente, strumenti che consentono di automatizzare la \textit{build} locale, un \textit{debugger} e 
                    strumenti per l'esecuzione di test automatici.\\
                    {[\textit{fonte riportata in sitografia}]}}
}

\newglossaryentry{webapp}{
    name={Web App},
    text={web app},
    sort=webapp,
    description={
                    L'applicazione \textit{web}, o abbreviato \textit{web app}, nell'ambito dell'informatica e della programmazione, 
                    si riferisce alle applicazioni accessibili e fruibili attraverso il \textit{web}, quindi accessibili dall'utente 
                    tramite un \textit{browser web} con una connessione attiva
                }
}



% Define custom colors
\definecolor{hyperColor}{HTML}{0B3EE3}
\definecolor{tableGray}{HTML}{F5F5F7}
\definecolor{veryPeri}{HTML}{6667ab}

% Set line height
\linespread{1.5}

% Custom hyphenation rules
\hyphenation {
    data-base
    al-go-rithms
    soft-ware
}

% Images path
\graphicspath{{img/}}

% Force page color, as some editors set a grayish color as default
\pagecolor{white}

% Better spacing for footnotes
\setlength{\skip\footins}{5mm}
\setlength{\footnotesep}{5mm}

\setlength{\headheight}{14.5pt}
\addtolength{\topmargin}{-2.45pt}

% Add a subscript G to a glossary entry
\newcommand{\glox}{\textsubscript{\textbf{\textit{G}}}}

% Improvements the paragraph command
\titleformat{\paragraph}
{\normalfont\normalsize\bfseries}{\theparagraph}{1em}{}
\titlespacing*{\paragraph}
{0pt}{3.25ex plus 1ex minus .2ex}{1.5ex plus .2ex}

% Define use case environment
\newcounter{usecasecounter} % define a counter
\setcounter{usecasecounter}{0} % set the counter to some initial value
% Parameters
% #1: ID
% #2: Nome
\newenvironment{usecase}[2]{
    \renewcommand{\theusecasecounter}{\usecasename #1}  % this is where the display of the counter is overwritten/modified
    \refstepcounter{usecasecounter} % increment counter
    \vspace{2em}
    \par \noindent % start new paragraph
    {\normalsize \textbf{\usecasename #1: #2}} % display the title before the content of the environment is displayed
    \vspace{.5em}
}{
    \medskip
}
\newcommand{\usecasename}{UC}
\newcommand{\usecaseactors}[1]{\textbf{\\Attori Principali:} #1}
\newcommand{\usecasepre}[1]{\textbf{\\Precondizioni:} #1}
\newcommand{\usecasedesc}[1]{\textbf{\\Descrizione:} #1}
\newcommand{\usecasepost}[1]{\textbf{\\Postcondizioni:} #1}
\newcommand{\usecasescen}[1]{\textbf{\\Scenario:} #1}

% Define risks environment
\newcounter{riskcounter} % define a counter
\setcounter{riskcounter}{0} % set the counter to some initial value
% Parameters
% #1: Title
\newenvironment{risk}[1]{
    \refstepcounter{riskcounter} % increment counter
    \par \noindent % start new paragraph
    \textbf{\arabic{riskcounter}. #1} % display the title before the content of the environment is displayed
}{
    \par\medskip
}
\newcommand{\riskname}{Rischio}
\newcommand{\riskdescription}[1]{\textbf{\\Descrizione:} #1.}
\newcommand{\risksolution}[1]{\textbf{\\Soluzione:} #1.}

% Apply fancy styling to pages
\pagestyle{fancy}
\fancyhf{}
\fancyhead[L]{\leftmark} % Places Chapter N. Chatper title on the top left
\fancyfoot[C]{\thepage} % Page number in the center of the footer

% Adds a blank page while increasing the page number
\newcommand\blankpage{ 
\clearpage
    \begingroup
    \null
    \thispagestyle{empty}
    \hypersetup{pageanchor=false}
    \clearpage
\endgroup
}

% Adds a blank page while increasing the page number
\newcommand\blankpagewithnumber{ 
  \clearpage
  \thispagestyle{plain} % Use plain page style to keep the page number
  \null
  \clearpage
}

% Increase page numbering
\newcommand\increasepagenumbering{
    \addtocounter{page}{+1}
}

% Make glossaries and bibliography
\makeglossaries
% Redefine the format for the glossary entries to be italic
\renewcommand*{\glstextformat}[1]{\textit{#1}\glox}
%\glsaddall

\bibliography{references/bibliography}
\defbibheading{bibliography} {
    \cleardoublepage
    \phantomsection
    \addcontentsline{toc}{chapter}{\bibname}
    \chapter*{\bibname\markboth{\bibname}{\bibname}}
}

% Code blocks handling w/ table of codes
\makeatletter
\ifdefined\NR@chapter
  \expandafter\@firstoftwo
\else
  \expandafter\@secondoftwo
\fi{\patchcmd\NR@chapter}{\patchcmd\@chapter}
  {\addtocontents{lot}{\protect\addvspace{10\p@}}}
  {\addtocontents{lot}{\protect\addvspace{10\p@}}%
   \addtocontents{lol}{\protect\addvspace{10\p@}}}
  {}{}
\makeatother

\renewcommand\listingscaption{Codice}
\renewcommand\listoflistingscaption{Elenco dei codici sorgenti}
\counterwithin*{listing}{chapter}
\renewcommand\thelisting{\thechapter.\arabic{listing}}

% Set up hyperlinks
\hypersetup{
    colorlinks=true,
    linktocpage=true,
    pdfstartpage=1,
    pdfstartview=,
    breaklinks=true,
    pdfpagemode=UseNone,
    pageanchor=true,
    pdfpagemode=UseOutlines,
    plainpages=false,
    bookmarksnumbered,
    bookmarksopen=true,
    bookmarksopenlevel=1,
    hypertexnames=true,
    pdfhighlight=/O,
    allcolors = hyperColor
}

% Set up captions
\captionsetup{
    tableposition=top,
    figureposition=bottom,
    font=small,
    format=hang,
    labelfont=bf
}

% When draft mode is on, the hyperlinks are removed. Useful when printing the document. To enable/disable, uncomment/comment the command
% \hypersetup{draft}

% prevents cleaning up the cache at the end of the run (needed to keep the unused caches, generated by other editions)
\makeatletter
\renewcommand*{\minted@cleancache}{}
\makeatother

% Break lines in code blocks whe using inputminted
\setminted{breaklines}